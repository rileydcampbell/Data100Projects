
% Default to the notebook output style

    


% Inherit from the specified cell style.




    
\documentclass[11pt]{article}

    
    
    \usepackage[T1]{fontenc}
    % Nicer default font (+ math font) than Computer Modern for most use cases
    \usepackage{mathpazo}

    % Basic figure setup, for now with no caption control since it's done
    % automatically by Pandoc (which extracts ![](path) syntax from Markdown).
    \usepackage{graphicx}
    % We will generate all images so they have a width \maxwidth. This means
    % that they will get their normal width if they fit onto the page, but
    % are scaled down if they would overflow the margins.
    \makeatletter
    \def\maxwidth{\ifdim\Gin@nat@width>\linewidth\linewidth
    \else\Gin@nat@width\fi}
    \makeatother
    \let\Oldincludegraphics\includegraphics
    % Set max figure width to be 80% of text width, for now hardcoded.
    \renewcommand{\includegraphics}[1]{\Oldincludegraphics[width=.8\maxwidth]{#1}}
    % Ensure that by default, figures have no caption (until we provide a
    % proper Figure object with a Caption API and a way to capture that
    % in the conversion process - todo).
    \usepackage{caption}
    \DeclareCaptionLabelFormat{nolabel}{}
    \captionsetup{labelformat=nolabel}

    \usepackage{adjustbox} % Used to constrain images to a maximum size 
    \usepackage{xcolor} % Allow colors to be defined
    \usepackage{enumerate} % Needed for markdown enumerations to work
    \usepackage{geometry} % Used to adjust the document margins
    \usepackage{amsmath} % Equations
    \usepackage{amssymb} % Equations
    \usepackage{textcomp} % defines textquotesingle
    % Hack from http://tex.stackexchange.com/a/47451/13684:
    \AtBeginDocument{%
        \def\PYZsq{\textquotesingle}% Upright quotes in Pygmentized code
    }
    \usepackage{upquote} % Upright quotes for verbatim code
    \usepackage{eurosym} % defines \euro
    \usepackage[mathletters]{ucs} % Extended unicode (utf-8) support
    \usepackage[utf8x]{inputenc} % Allow utf-8 characters in the tex document
    \usepackage{fancyvrb} % verbatim replacement that allows latex
    \usepackage{grffile} % extends the file name processing of package graphics 
                         % to support a larger range 
    % The hyperref package gives us a pdf with properly built
    % internal navigation ('pdf bookmarks' for the table of contents,
    % internal cross-reference links, web links for URLs, etc.)
    \usepackage{hyperref}
    \usepackage{longtable} % longtable support required by pandoc >1.10
    \usepackage{booktabs}  % table support for pandoc > 1.12.2
    \usepackage[inline]{enumitem} % IRkernel/repr support (it uses the enumerate* environment)
    \usepackage[normalem]{ulem} % ulem is needed to support strikethroughs (\sout)
                                % normalem makes italics be italics, not underlines
    

    
    
    % Colors for the hyperref package
    \definecolor{urlcolor}{rgb}{0,.145,.698}
    \definecolor{linkcolor}{rgb}{.71,0.21,0.01}
    \definecolor{citecolor}{rgb}{.12,.54,.11}

    % ANSI colors
    \definecolor{ansi-black}{HTML}{3E424D}
    \definecolor{ansi-black-intense}{HTML}{282C36}
    \definecolor{ansi-red}{HTML}{E75C58}
    \definecolor{ansi-red-intense}{HTML}{B22B31}
    \definecolor{ansi-green}{HTML}{00A250}
    \definecolor{ansi-green-intense}{HTML}{007427}
    \definecolor{ansi-yellow}{HTML}{DDB62B}
    \definecolor{ansi-yellow-intense}{HTML}{B27D12}
    \definecolor{ansi-blue}{HTML}{208FFB}
    \definecolor{ansi-blue-intense}{HTML}{0065CA}
    \definecolor{ansi-magenta}{HTML}{D160C4}
    \definecolor{ansi-magenta-intense}{HTML}{A03196}
    \definecolor{ansi-cyan}{HTML}{60C6C8}
    \definecolor{ansi-cyan-intense}{HTML}{258F8F}
    \definecolor{ansi-white}{HTML}{C5C1B4}
    \definecolor{ansi-white-intense}{HTML}{A1A6B2}

    % commands and environments needed by pandoc snippets
    % extracted from the output of `pandoc -s`
    \providecommand{\tightlist}{%
      \setlength{\itemsep}{0pt}\setlength{\parskip}{0pt}}
    \DefineVerbatimEnvironment{Highlighting}{Verbatim}{commandchars=\\\{\}}
    % Add ',fontsize=\small' for more characters per line
    \newenvironment{Shaded}{}{}
    \newcommand{\KeywordTok}[1]{\textcolor[rgb]{0.00,0.44,0.13}{\textbf{{#1}}}}
    \newcommand{\DataTypeTok}[1]{\textcolor[rgb]{0.56,0.13,0.00}{{#1}}}
    \newcommand{\DecValTok}[1]{\textcolor[rgb]{0.25,0.63,0.44}{{#1}}}
    \newcommand{\BaseNTok}[1]{\textcolor[rgb]{0.25,0.63,0.44}{{#1}}}
    \newcommand{\FloatTok}[1]{\textcolor[rgb]{0.25,0.63,0.44}{{#1}}}
    \newcommand{\CharTok}[1]{\textcolor[rgb]{0.25,0.44,0.63}{{#1}}}
    \newcommand{\StringTok}[1]{\textcolor[rgb]{0.25,0.44,0.63}{{#1}}}
    \newcommand{\CommentTok}[1]{\textcolor[rgb]{0.38,0.63,0.69}{\textit{{#1}}}}
    \newcommand{\OtherTok}[1]{\textcolor[rgb]{0.00,0.44,0.13}{{#1}}}
    \newcommand{\AlertTok}[1]{\textcolor[rgb]{1.00,0.00,0.00}{\textbf{{#1}}}}
    \newcommand{\FunctionTok}[1]{\textcolor[rgb]{0.02,0.16,0.49}{{#1}}}
    \newcommand{\RegionMarkerTok}[1]{{#1}}
    \newcommand{\ErrorTok}[1]{\textcolor[rgb]{1.00,0.00,0.00}{\textbf{{#1}}}}
    \newcommand{\NormalTok}[1]{{#1}}
    
    % Additional commands for more recent versions of Pandoc
    \newcommand{\ConstantTok}[1]{\textcolor[rgb]{0.53,0.00,0.00}{{#1}}}
    \newcommand{\SpecialCharTok}[1]{\textcolor[rgb]{0.25,0.44,0.63}{{#1}}}
    \newcommand{\VerbatimStringTok}[1]{\textcolor[rgb]{0.25,0.44,0.63}{{#1}}}
    \newcommand{\SpecialStringTok}[1]{\textcolor[rgb]{0.73,0.40,0.53}{{#1}}}
    \newcommand{\ImportTok}[1]{{#1}}
    \newcommand{\DocumentationTok}[1]{\textcolor[rgb]{0.73,0.13,0.13}{\textit{{#1}}}}
    \newcommand{\AnnotationTok}[1]{\textcolor[rgb]{0.38,0.63,0.69}{\textbf{\textit{{#1}}}}}
    \newcommand{\CommentVarTok}[1]{\textcolor[rgb]{0.38,0.63,0.69}{\textbf{\textit{{#1}}}}}
    \newcommand{\VariableTok}[1]{\textcolor[rgb]{0.10,0.09,0.49}{{#1}}}
    \newcommand{\ControlFlowTok}[1]{\textcolor[rgb]{0.00,0.44,0.13}{\textbf{{#1}}}}
    \newcommand{\OperatorTok}[1]{\textcolor[rgb]{0.40,0.40,0.40}{{#1}}}
    \newcommand{\BuiltInTok}[1]{{#1}}
    \newcommand{\ExtensionTok}[1]{{#1}}
    \newcommand{\PreprocessorTok}[1]{\textcolor[rgb]{0.74,0.48,0.00}{{#1}}}
    \newcommand{\AttributeTok}[1]{\textcolor[rgb]{0.49,0.56,0.16}{{#1}}}
    \newcommand{\InformationTok}[1]{\textcolor[rgb]{0.38,0.63,0.69}{\textbf{\textit{{#1}}}}}
    \newcommand{\WarningTok}[1]{\textcolor[rgb]{0.38,0.63,0.69}{\textbf{\textit{{#1}}}}}
    
    
    % Define a nice break command that doesn't care if a line doesn't already
    % exist.
    \def\br{\hspace*{\fill} \\* }
    % Math Jax compatability definitions
    \def\gt{>}
    \def\lt{<}
    % Document parameters
    \title{hw2}
    
    
    

    % Pygments definitions
    
\makeatletter
\def\PY@reset{\let\PY@it=\relax \let\PY@bf=\relax%
    \let\PY@ul=\relax \let\PY@tc=\relax%
    \let\PY@bc=\relax \let\PY@ff=\relax}
\def\PY@tok#1{\csname PY@tok@#1\endcsname}
\def\PY@toks#1+{\ifx\relax#1\empty\else%
    \PY@tok{#1}\expandafter\PY@toks\fi}
\def\PY@do#1{\PY@bc{\PY@tc{\PY@ul{%
    \PY@it{\PY@bf{\PY@ff{#1}}}}}}}
\def\PY#1#2{\PY@reset\PY@toks#1+\relax+\PY@do{#2}}

\expandafter\def\csname PY@tok@w\endcsname{\def\PY@tc##1{\textcolor[rgb]{0.73,0.73,0.73}{##1}}}
\expandafter\def\csname PY@tok@c\endcsname{\let\PY@it=\textit\def\PY@tc##1{\textcolor[rgb]{0.25,0.50,0.50}{##1}}}
\expandafter\def\csname PY@tok@cp\endcsname{\def\PY@tc##1{\textcolor[rgb]{0.74,0.48,0.00}{##1}}}
\expandafter\def\csname PY@tok@k\endcsname{\let\PY@bf=\textbf\def\PY@tc##1{\textcolor[rgb]{0.00,0.50,0.00}{##1}}}
\expandafter\def\csname PY@tok@kp\endcsname{\def\PY@tc##1{\textcolor[rgb]{0.00,0.50,0.00}{##1}}}
\expandafter\def\csname PY@tok@kt\endcsname{\def\PY@tc##1{\textcolor[rgb]{0.69,0.00,0.25}{##1}}}
\expandafter\def\csname PY@tok@o\endcsname{\def\PY@tc##1{\textcolor[rgb]{0.40,0.40,0.40}{##1}}}
\expandafter\def\csname PY@tok@ow\endcsname{\let\PY@bf=\textbf\def\PY@tc##1{\textcolor[rgb]{0.67,0.13,1.00}{##1}}}
\expandafter\def\csname PY@tok@nb\endcsname{\def\PY@tc##1{\textcolor[rgb]{0.00,0.50,0.00}{##1}}}
\expandafter\def\csname PY@tok@nf\endcsname{\def\PY@tc##1{\textcolor[rgb]{0.00,0.00,1.00}{##1}}}
\expandafter\def\csname PY@tok@nc\endcsname{\let\PY@bf=\textbf\def\PY@tc##1{\textcolor[rgb]{0.00,0.00,1.00}{##1}}}
\expandafter\def\csname PY@tok@nn\endcsname{\let\PY@bf=\textbf\def\PY@tc##1{\textcolor[rgb]{0.00,0.00,1.00}{##1}}}
\expandafter\def\csname PY@tok@ne\endcsname{\let\PY@bf=\textbf\def\PY@tc##1{\textcolor[rgb]{0.82,0.25,0.23}{##1}}}
\expandafter\def\csname PY@tok@nv\endcsname{\def\PY@tc##1{\textcolor[rgb]{0.10,0.09,0.49}{##1}}}
\expandafter\def\csname PY@tok@no\endcsname{\def\PY@tc##1{\textcolor[rgb]{0.53,0.00,0.00}{##1}}}
\expandafter\def\csname PY@tok@nl\endcsname{\def\PY@tc##1{\textcolor[rgb]{0.63,0.63,0.00}{##1}}}
\expandafter\def\csname PY@tok@ni\endcsname{\let\PY@bf=\textbf\def\PY@tc##1{\textcolor[rgb]{0.60,0.60,0.60}{##1}}}
\expandafter\def\csname PY@tok@na\endcsname{\def\PY@tc##1{\textcolor[rgb]{0.49,0.56,0.16}{##1}}}
\expandafter\def\csname PY@tok@nt\endcsname{\let\PY@bf=\textbf\def\PY@tc##1{\textcolor[rgb]{0.00,0.50,0.00}{##1}}}
\expandafter\def\csname PY@tok@nd\endcsname{\def\PY@tc##1{\textcolor[rgb]{0.67,0.13,1.00}{##1}}}
\expandafter\def\csname PY@tok@s\endcsname{\def\PY@tc##1{\textcolor[rgb]{0.73,0.13,0.13}{##1}}}
\expandafter\def\csname PY@tok@sd\endcsname{\let\PY@it=\textit\def\PY@tc##1{\textcolor[rgb]{0.73,0.13,0.13}{##1}}}
\expandafter\def\csname PY@tok@si\endcsname{\let\PY@bf=\textbf\def\PY@tc##1{\textcolor[rgb]{0.73,0.40,0.53}{##1}}}
\expandafter\def\csname PY@tok@se\endcsname{\let\PY@bf=\textbf\def\PY@tc##1{\textcolor[rgb]{0.73,0.40,0.13}{##1}}}
\expandafter\def\csname PY@tok@sr\endcsname{\def\PY@tc##1{\textcolor[rgb]{0.73,0.40,0.53}{##1}}}
\expandafter\def\csname PY@tok@ss\endcsname{\def\PY@tc##1{\textcolor[rgb]{0.10,0.09,0.49}{##1}}}
\expandafter\def\csname PY@tok@sx\endcsname{\def\PY@tc##1{\textcolor[rgb]{0.00,0.50,0.00}{##1}}}
\expandafter\def\csname PY@tok@m\endcsname{\def\PY@tc##1{\textcolor[rgb]{0.40,0.40,0.40}{##1}}}
\expandafter\def\csname PY@tok@gh\endcsname{\let\PY@bf=\textbf\def\PY@tc##1{\textcolor[rgb]{0.00,0.00,0.50}{##1}}}
\expandafter\def\csname PY@tok@gu\endcsname{\let\PY@bf=\textbf\def\PY@tc##1{\textcolor[rgb]{0.50,0.00,0.50}{##1}}}
\expandafter\def\csname PY@tok@gd\endcsname{\def\PY@tc##1{\textcolor[rgb]{0.63,0.00,0.00}{##1}}}
\expandafter\def\csname PY@tok@gi\endcsname{\def\PY@tc##1{\textcolor[rgb]{0.00,0.63,0.00}{##1}}}
\expandafter\def\csname PY@tok@gr\endcsname{\def\PY@tc##1{\textcolor[rgb]{1.00,0.00,0.00}{##1}}}
\expandafter\def\csname PY@tok@ge\endcsname{\let\PY@it=\textit}
\expandafter\def\csname PY@tok@gs\endcsname{\let\PY@bf=\textbf}
\expandafter\def\csname PY@tok@gp\endcsname{\let\PY@bf=\textbf\def\PY@tc##1{\textcolor[rgb]{0.00,0.00,0.50}{##1}}}
\expandafter\def\csname PY@tok@go\endcsname{\def\PY@tc##1{\textcolor[rgb]{0.53,0.53,0.53}{##1}}}
\expandafter\def\csname PY@tok@gt\endcsname{\def\PY@tc##1{\textcolor[rgb]{0.00,0.27,0.87}{##1}}}
\expandafter\def\csname PY@tok@err\endcsname{\def\PY@bc##1{\setlength{\fboxsep}{0pt}\fcolorbox[rgb]{1.00,0.00,0.00}{1,1,1}{\strut ##1}}}
\expandafter\def\csname PY@tok@kc\endcsname{\let\PY@bf=\textbf\def\PY@tc##1{\textcolor[rgb]{0.00,0.50,0.00}{##1}}}
\expandafter\def\csname PY@tok@kd\endcsname{\let\PY@bf=\textbf\def\PY@tc##1{\textcolor[rgb]{0.00,0.50,0.00}{##1}}}
\expandafter\def\csname PY@tok@kn\endcsname{\let\PY@bf=\textbf\def\PY@tc##1{\textcolor[rgb]{0.00,0.50,0.00}{##1}}}
\expandafter\def\csname PY@tok@kr\endcsname{\let\PY@bf=\textbf\def\PY@tc##1{\textcolor[rgb]{0.00,0.50,0.00}{##1}}}
\expandafter\def\csname PY@tok@bp\endcsname{\def\PY@tc##1{\textcolor[rgb]{0.00,0.50,0.00}{##1}}}
\expandafter\def\csname PY@tok@fm\endcsname{\def\PY@tc##1{\textcolor[rgb]{0.00,0.00,1.00}{##1}}}
\expandafter\def\csname PY@tok@vc\endcsname{\def\PY@tc##1{\textcolor[rgb]{0.10,0.09,0.49}{##1}}}
\expandafter\def\csname PY@tok@vg\endcsname{\def\PY@tc##1{\textcolor[rgb]{0.10,0.09,0.49}{##1}}}
\expandafter\def\csname PY@tok@vi\endcsname{\def\PY@tc##1{\textcolor[rgb]{0.10,0.09,0.49}{##1}}}
\expandafter\def\csname PY@tok@vm\endcsname{\def\PY@tc##1{\textcolor[rgb]{0.10,0.09,0.49}{##1}}}
\expandafter\def\csname PY@tok@sa\endcsname{\def\PY@tc##1{\textcolor[rgb]{0.73,0.13,0.13}{##1}}}
\expandafter\def\csname PY@tok@sb\endcsname{\def\PY@tc##1{\textcolor[rgb]{0.73,0.13,0.13}{##1}}}
\expandafter\def\csname PY@tok@sc\endcsname{\def\PY@tc##1{\textcolor[rgb]{0.73,0.13,0.13}{##1}}}
\expandafter\def\csname PY@tok@dl\endcsname{\def\PY@tc##1{\textcolor[rgb]{0.73,0.13,0.13}{##1}}}
\expandafter\def\csname PY@tok@s2\endcsname{\def\PY@tc##1{\textcolor[rgb]{0.73,0.13,0.13}{##1}}}
\expandafter\def\csname PY@tok@sh\endcsname{\def\PY@tc##1{\textcolor[rgb]{0.73,0.13,0.13}{##1}}}
\expandafter\def\csname PY@tok@s1\endcsname{\def\PY@tc##1{\textcolor[rgb]{0.73,0.13,0.13}{##1}}}
\expandafter\def\csname PY@tok@mb\endcsname{\def\PY@tc##1{\textcolor[rgb]{0.40,0.40,0.40}{##1}}}
\expandafter\def\csname PY@tok@mf\endcsname{\def\PY@tc##1{\textcolor[rgb]{0.40,0.40,0.40}{##1}}}
\expandafter\def\csname PY@tok@mh\endcsname{\def\PY@tc##1{\textcolor[rgb]{0.40,0.40,0.40}{##1}}}
\expandafter\def\csname PY@tok@mi\endcsname{\def\PY@tc##1{\textcolor[rgb]{0.40,0.40,0.40}{##1}}}
\expandafter\def\csname PY@tok@il\endcsname{\def\PY@tc##1{\textcolor[rgb]{0.40,0.40,0.40}{##1}}}
\expandafter\def\csname PY@tok@mo\endcsname{\def\PY@tc##1{\textcolor[rgb]{0.40,0.40,0.40}{##1}}}
\expandafter\def\csname PY@tok@ch\endcsname{\let\PY@it=\textit\def\PY@tc##1{\textcolor[rgb]{0.25,0.50,0.50}{##1}}}
\expandafter\def\csname PY@tok@cm\endcsname{\let\PY@it=\textit\def\PY@tc##1{\textcolor[rgb]{0.25,0.50,0.50}{##1}}}
\expandafter\def\csname PY@tok@cpf\endcsname{\let\PY@it=\textit\def\PY@tc##1{\textcolor[rgb]{0.25,0.50,0.50}{##1}}}
\expandafter\def\csname PY@tok@c1\endcsname{\let\PY@it=\textit\def\PY@tc##1{\textcolor[rgb]{0.25,0.50,0.50}{##1}}}
\expandafter\def\csname PY@tok@cs\endcsname{\let\PY@it=\textit\def\PY@tc##1{\textcolor[rgb]{0.25,0.50,0.50}{##1}}}

\def\PYZbs{\char`\\}
\def\PYZus{\char`\_}
\def\PYZob{\char`\{}
\def\PYZcb{\char`\}}
\def\PYZca{\char`\^}
\def\PYZam{\char`\&}
\def\PYZlt{\char`\<}
\def\PYZgt{\char`\>}
\def\PYZsh{\char`\#}
\def\PYZpc{\char`\%}
\def\PYZdl{\char`\$}
\def\PYZhy{\char`\-}
\def\PYZsq{\char`\'}
\def\PYZdq{\char`\"}
\def\PYZti{\char`\~}
% for compatibility with earlier versions
\def\PYZat{@}
\def\PYZlb{[}
\def\PYZrb{]}
\makeatother


    % Exact colors from NB
    \definecolor{incolor}{rgb}{0.0, 0.0, 0.5}
    \definecolor{outcolor}{rgb}{0.545, 0.0, 0.0}



    
    % Prevent overflowing lines due to hard-to-break entities
    \sloppy 
    % Setup hyperref package
    \hypersetup{
      breaklinks=true,  % so long urls are correctly broken across lines
      colorlinks=true,
      urlcolor=urlcolor,
      linkcolor=linkcolor,
      citecolor=citecolor,
      }
    % Slightly bigger margins than the latex defaults
    
    \geometry{verbose,tmargin=1in,bmargin=1in,lmargin=1in,rmargin=1in}
    
    

    \begin{document}
    
    
    \maketitle
    
    

    
    Before you turn this problem in, make sure everything runs as expected.
First, \textbf{restart the kernel} (in the menubar, select
Kernel\(\rightarrow\)Restart) and then \textbf{run all cells} (in the
menubar, select Cell\(\rightarrow\)Run All).

Make sure you fill in any place that says \texttt{YOUR\ CODE\ HERE} or
"YOUR ANSWER HERE", as well as your name and collaborators below:

    \begin{Verbatim}[commandchars=\\\{\}]
{\color{incolor}In [{\color{incolor}1}]:} \PY{n}{NAME} \PY{o}{=} \PY{l+s+s2}{\PYZdq{}}\PY{l+s+s2}{Riley Campbell}\PY{l+s+s2}{\PYZdq{}}
        \PY{n}{COLLABORATORS} \PY{o}{=} \PY{l+s+s2}{\PYZdq{}}\PY{l+s+s2}{\PYZdq{}}
\end{Verbatim}


    \begin{center}\rule{0.5\linewidth}{\linethickness}\end{center}

    \section{Homework 2: Bike Sharing}\label{homework-2-bike-sharing}

\subsection{EDA and Visualization}\label{eda-and-visualization}

\subsection{Due Date: Tuesday 9/25, 11:59
PM}\label{due-date-tuesday-925-1159-pm}

\subsection{Course Policies}\label{course-policies}

Here are some important course policies. These are also located at
http://www.ds100.org/fa18/.

\textbf{Collaboration Policy}

Data science is a collaborative activity. While you may talk with others
about the homework, we ask that you \textbf{write your solutions
individually}. If you do discuss the assignments with others please
\textbf{include their names} at the top of your solution.

\subsection{Introduction}\label{introduction}

This assignment includes both specific tasks to perform and open-ended
questions to investigate. The open-ended questions ask you to think
creatively and critically about how the plots you have created provide
insight into the data.

After completing this assignment, you should be comfortable with:

\begin{itemize}
\tightlist
\item
  reading plaintext delimited data into \texttt{pandas}
\item
  wrangling data for analysis
\item
  using EDA to learn about your data
\item
  making informative plots
\end{itemize}

\subsection{Grading}\label{grading}

Grading is broken down into autograded answers and free response.

For autograded answers, the results of your code are compared to public
and/or hidden tests.

For free response, readers will evaluate how well you answered the
question and/or fulfilled the requirements of the question.

For plots, your plots should be \emph{similar} to the given examples. We
will tolerate small variations such as color differences or slight
variations in scale. However it is in your best interest to make the
plots as similar as possible as similarity is subject to the readers.

\textbf{Note that for ALL plotting questions from here on out, we will
expect appropriate titles, axis labels, legends, etc. The following
question serves as a good guideline on what is "enough": If I directly
downloaded the plot and viewed it, would I be able to tell what was
being visualized without knowing the question?}

\subsection{Submission - IMPORTANT, PLEASE
READ}\label{submission---important-please-read}

For this assignment and future assignments (homework and projects) you
will also submit your free response and plotting questions to
gradescope. To do this, you can download as PDF
(\texttt{File-\textgreater{}Download\ As-\textgreater{}PDF\ via\ Latex\ (.pdf)}).
You are responsible for submitting and tagging your answers in
gradescope. For each free response and plotting question, please
include:

\begin{enumerate}
\def\labelenumi{\arabic{enumi}.}
\tightlist
\item
  Relevant code used to generate the plot or inform your insights
\item
  The written free response or plot
\end{enumerate}

We are doing this to make it easier on our graders and for you, in the
case you need to submit a regrade request. Gradescope (as of now) is
still better for manual grading.

\subsubsection{Score breakdown}\label{score-breakdown}

\begin{longtable}[]{@{}ll@{}}
\toprule
Question & Points\tabularnewline
\midrule
\endhead
Question 1a & 2\tabularnewline
Question 1b & 1\tabularnewline
Question 1c & 2\tabularnewline
Question 2a & 2\tabularnewline
Question 2b & 2\tabularnewline
Question 2c & 2\tabularnewline
Question 2d & 2\tabularnewline
Question 3a & 5\tabularnewline
Question 3b & 3\tabularnewline
Question 3c & 2\tabularnewline
Question 4a & 4\tabularnewline
Question 4b & 3\tabularnewline
Question 5a & 2\tabularnewline
Question 5b & 2\tabularnewline
Question 6a & 1\tabularnewline
Question 6b & 4\tabularnewline
Question 6c & 2\tabularnewline
Total & 41\tabularnewline
\bottomrule
\end{longtable}

    \begin{Verbatim}[commandchars=\\\{\}]
{\color{incolor}In [{\color{incolor}2}]:} \PY{c+c1}{\PYZsh{} Run this cell to set up your notebook.  Make sure ds100\PYZus{}utils.py is in this assignment\PYZsq{}s folder}
        \PY{k+kn}{import} \PY{n+nn}{seaborn} \PY{k}{as} \PY{n+nn}{sns}
        \PY{k+kn}{import} \PY{n+nn}{csv}
        \PY{k+kn}{import} \PY{n+nn}{numpy} \PY{k}{as} \PY{n+nn}{np}
        \PY{k+kn}{import} \PY{n+nn}{pandas} \PY{k}{as} \PY{n+nn}{pd}
        \PY{k+kn}{import} \PY{n+nn}{matplotlib}\PY{n+nn}{.}\PY{n+nn}{pyplot} \PY{k}{as} \PY{n+nn}{plt}
        \PY{k+kn}{import} \PY{n+nn}{zipfile}
        \PY{k+kn}{from} \PY{n+nn}{pathlib} \PY{k}{import} \PY{n}{Path}
        \PY{k+kn}{import} \PY{n+nn}{ds100\PYZus{}utils}
        
        \PY{c+c1}{\PYZsh{} Default plot configurations}
        \PY{o}{\PYZpc{}}\PY{k}{matplotlib} inline
        \PY{n}{plt}\PY{o}{.}\PY{n}{rcParams}\PY{p}{[}\PY{l+s+s1}{\PYZsq{}}\PY{l+s+s1}{figure.figsize}\PY{l+s+s1}{\PYZsq{}}\PY{p}{]} \PY{o}{=} \PY{p}{(}\PY{l+m+mi}{16}\PY{p}{,}\PY{l+m+mi}{8}\PY{p}{)}
        \PY{n}{plt}\PY{o}{.}\PY{n}{rcParams}\PY{p}{[}\PY{l+s+s1}{\PYZsq{}}\PY{l+s+s1}{figure.dpi}\PY{l+s+s1}{\PYZsq{}}\PY{p}{]} \PY{o}{=} \PY{l+m+mi}{150}
        \PY{n}{sns}\PY{o}{.}\PY{n}{set}\PY{p}{(}\PY{p}{)}
        
        \PY{k+kn}{from} \PY{n+nn}{IPython}\PY{n+nn}{.}\PY{n+nn}{display} \PY{k}{import} \PY{n}{display}\PY{p}{,} \PY{n}{Latex}\PY{p}{,} \PY{n}{Markdown}
\end{Verbatim}


    \subsection{Loading Bike Sharing Data}\label{loading-bike-sharing-data}

The data we are exploring is data on bike sharing in Washington D.C.

The variables in this data frame are defined as:

\begin{longtable}[]{@{}ll@{}}
\toprule
\begin{minipage}[b]{0.17\columnwidth}\raggedright\strut
Variable\strut
\end{minipage} & \begin{minipage}[b]{0.77\columnwidth}\raggedright\strut
Description\strut
\end{minipage}\tabularnewline
\midrule
\endhead
\begin{minipage}[t]{0.17\columnwidth}\raggedright\strut
instant\strut
\end{minipage} & \begin{minipage}[t]{0.77\columnwidth}\raggedright\strut
record index\strut
\end{minipage}\tabularnewline
\begin{minipage}[t]{0.17\columnwidth}\raggedright\strut
dteday\strut
\end{minipage} & \begin{minipage}[t]{0.77\columnwidth}\raggedright\strut
date\strut
\end{minipage}\tabularnewline
\begin{minipage}[t]{0.17\columnwidth}\raggedright\strut
season\strut
\end{minipage} & \begin{minipage}[t]{0.77\columnwidth}\raggedright\strut
1. spring 2. summer 3. fall 4. winter\strut
\end{minipage}\tabularnewline
\begin{minipage}[t]{0.17\columnwidth}\raggedright\strut
yr\strut
\end{minipage} & \begin{minipage}[t]{0.77\columnwidth}\raggedright\strut
year (0: 2011, 1:2012)\strut
\end{minipage}\tabularnewline
\begin{minipage}[t]{0.17\columnwidth}\raggedright\strut
mnth\strut
\end{minipage} & \begin{minipage}[t]{0.77\columnwidth}\raggedright\strut
month ( 1 to 12)\strut
\end{minipage}\tabularnewline
\begin{minipage}[t]{0.17\columnwidth}\raggedright\strut
hr\strut
\end{minipage} & \begin{minipage}[t]{0.77\columnwidth}\raggedright\strut
hour (0 to 23)\strut
\end{minipage}\tabularnewline
\begin{minipage}[t]{0.17\columnwidth}\raggedright\strut
holiday\strut
\end{minipage} & \begin{minipage}[t]{0.77\columnwidth}\raggedright\strut
whether day is holiday or not\strut
\end{minipage}\tabularnewline
\begin{minipage}[t]{0.17\columnwidth}\raggedright\strut
weekday\strut
\end{minipage} & \begin{minipage}[t]{0.77\columnwidth}\raggedright\strut
day of the week\strut
\end{minipage}\tabularnewline
\begin{minipage}[t]{0.17\columnwidth}\raggedright\strut
workingday\strut
\end{minipage} & \begin{minipage}[t]{0.77\columnwidth}\raggedright\strut
if day is neither weekend nor holiday\strut
\end{minipage}\tabularnewline
\begin{minipage}[t]{0.17\columnwidth}\raggedright\strut
weathersit\strut
\end{minipage} & \begin{minipage}[t]{0.77\columnwidth}\raggedright\strut
1. clear or partly cloudy 2. mist and clouds 3. light snow or rain 4.
heavy rain or snow\strut
\end{minipage}\tabularnewline
\begin{minipage}[t]{0.17\columnwidth}\raggedright\strut
temp\strut
\end{minipage} & \begin{minipage}[t]{0.77\columnwidth}\raggedright\strut
normalized temperature in Celsius (divided by 41)\strut
\end{minipage}\tabularnewline
\begin{minipage}[t]{0.17\columnwidth}\raggedright\strut
atemp\strut
\end{minipage} & \begin{minipage}[t]{0.77\columnwidth}\raggedright\strut
normalized "feels-like" temperature in Celsius (divided by 50)\strut
\end{minipage}\tabularnewline
\begin{minipage}[t]{0.17\columnwidth}\raggedright\strut
hum\strut
\end{minipage} & \begin{minipage}[t]{0.77\columnwidth}\raggedright\strut
normalized percent humidity (divided by 100)\strut
\end{minipage}\tabularnewline
\begin{minipage}[t]{0.17\columnwidth}\raggedright\strut
windspeed\strut
\end{minipage} & \begin{minipage}[t]{0.77\columnwidth}\raggedright\strut
normalized wind speed (divided by 67)\strut
\end{minipage}\tabularnewline
\begin{minipage}[t]{0.17\columnwidth}\raggedright\strut
casual\strut
\end{minipage} & \begin{minipage}[t]{0.77\columnwidth}\raggedright\strut
count of casual users\strut
\end{minipage}\tabularnewline
\begin{minipage}[t]{0.17\columnwidth}\raggedright\strut
registered\strut
\end{minipage} & \begin{minipage}[t]{0.77\columnwidth}\raggedright\strut
count of registered users\strut
\end{minipage}\tabularnewline
\begin{minipage}[t]{0.17\columnwidth}\raggedright\strut
cnt\strut
\end{minipage} & \begin{minipage}[t]{0.77\columnwidth}\raggedright\strut
count of total rental bikes including casual and registered\strut
\end{minipage}\tabularnewline
\bottomrule
\end{longtable}

    \subsubsection{Download the Data}\label{download-the-data}

    \begin{Verbatim}[commandchars=\\\{\}]
{\color{incolor}In [{\color{incolor}3}]:} \PY{c+c1}{\PYZsh{} Run this cell to download the data.  No further action is needed}
        
        \PY{n}{data\PYZus{}url} \PY{o}{=} \PY{l+s+s1}{\PYZsq{}}\PY{l+s+s1}{https://github.com/DS\PYZhy{}100/fa18/raw/gh\PYZhy{}pages/assets/datasets/hw2\PYZhy{}bikeshare.zip}\PY{l+s+s1}{\PYZsq{}}
        \PY{n}{file\PYZus{}name} \PY{o}{=} \PY{l+s+s1}{\PYZsq{}}\PY{l+s+s1}{data.zip}\PY{l+s+s1}{\PYZsq{}}
        \PY{n}{data\PYZus{}dir} \PY{o}{=} \PY{l+s+s1}{\PYZsq{}}\PY{l+s+s1}{.}\PY{l+s+s1}{\PYZsq{}}
        
        \PY{n}{dest\PYZus{}path} \PY{o}{=} \PY{n}{ds100\PYZus{}utils}\PY{o}{.}\PY{n}{fetch\PYZus{}and\PYZus{}cache}\PY{p}{(}\PY{n}{data\PYZus{}url}\PY{o}{=}\PY{n}{data\PYZus{}url}\PY{p}{,} \PY{n}{data\PYZus{}dir}\PY{o}{=}\PY{n}{data\PYZus{}dir}\PY{p}{,} \PY{n}{file}\PY{o}{=}\PY{n}{file\PYZus{}name}\PY{p}{)}
        \PY{n+nb}{print}\PY{p}{(}\PY{l+s+s1}{\PYZsq{}}\PY{l+s+s1}{Saved at }\PY{l+s+si}{\PYZob{}\PYZcb{}}\PY{l+s+s1}{\PYZsq{}}\PY{o}{.}\PY{n}{format}\PY{p}{(}\PY{n}{dest\PYZus{}path}\PY{p}{)}\PY{p}{)}
        
        \PY{n}{zipped\PYZus{}data} \PY{o}{=} \PY{n}{zipfile}\PY{o}{.}\PY{n}{ZipFile}\PY{p}{(}\PY{n}{dest\PYZus{}path}\PY{p}{,} \PY{l+s+s1}{\PYZsq{}}\PY{l+s+s1}{r}\PY{l+s+s1}{\PYZsq{}}\PY{p}{)}
        
        \PY{n}{data\PYZus{}dir} \PY{o}{=} \PY{n}{Path}\PY{p}{(}\PY{l+s+s1}{\PYZsq{}}\PY{l+s+s1}{data}\PY{l+s+s1}{\PYZsq{}}\PY{p}{)}
        \PY{n}{zipped\PYZus{}data}\PY{o}{.}\PY{n}{extractall}\PY{p}{(}\PY{n}{data\PYZus{}dir}\PY{p}{)}
        
        
        \PY{n+nb}{print}\PY{p}{(}\PY{l+s+s2}{\PYZdq{}}\PY{l+s+s2}{Extracted Files:}\PY{l+s+s2}{\PYZdq{}}\PY{p}{)}
        \PY{k}{for} \PY{n}{f} \PY{o+ow}{in} \PY{n}{data\PYZus{}dir}\PY{o}{.}\PY{n}{glob}\PY{p}{(}\PY{l+s+s2}{\PYZdq{}}\PY{l+s+s2}{*}\PY{l+s+s2}{\PYZdq{}}\PY{p}{)}\PY{p}{:}
            \PY{n+nb}{print}\PY{p}{(}\PY{l+s+s2}{\PYZdq{}}\PY{l+s+se}{\PYZbs{}t}\PY{l+s+s2}{\PYZdq{}}\PY{p}{,}\PY{n}{f}\PY{p}{)}
\end{Verbatim}


    \begin{Verbatim}[commandchars=\\\{\}]
Using version already downloaded: Wed Sep 19 03:06:12 2018
MD5 hash of file: 2bcd2ca89278a8230f4e9461455c0811
Saved at data.zip
Extracted Files:
	 data/bikeshare.txt

    \end{Verbatim}

    \subsubsection{Examining the file
contents}\label{examining-the-file-contents}

Can you identify the file format? (No answer required.)

    \begin{Verbatim}[commandchars=\\\{\}]
{\color{incolor}In [{\color{incolor}4}]:} \PY{c+c1}{\PYZsh{} Run this cell to look at the top of the file.  No further action is needed}
        \PY{k}{for} \PY{n}{line} \PY{o+ow}{in} \PY{n}{ds100\PYZus{}utils}\PY{o}{.}\PY{n}{head}\PY{p}{(}\PY{n}{data\PYZus{}dir}\PY{o}{/}\PY{l+s+s1}{\PYZsq{}}\PY{l+s+s1}{bikeshare.txt}\PY{l+s+s1}{\PYZsq{}}\PY{p}{)}\PY{p}{:}
            \PY{n+nb}{print}\PY{p}{(}\PY{n}{line}\PY{p}{,}\PY{n}{end}\PY{o}{=}\PY{l+s+s2}{\PYZdq{}}\PY{l+s+s2}{\PYZdq{}}\PY{p}{)}
\end{Verbatim}


    \begin{Verbatim}[commandchars=\\\{\}]
instant,dteday,season,yr,mnth,hr,holiday,weekday,workingday,weathersit,temp,atemp,hum,windspeed,casual,registered,cnt
1,2011-01-01,1,0,1,0,0,6,0,1,0.24,0.2879,0.81,0,3,13,16
2,2011-01-01,1,0,1,1,0,6,0,1,0.22,0.2727,0.8,0,8,32,40
3,2011-01-01,1,0,1,2,0,6,0,1,0.22,0.2727,0.8,0,5,27,32
4,2011-01-01,1,0,1,3,0,6,0,1,0.24,0.2879,0.75,0,3,10,13

    \end{Verbatim}

    \subsubsection{Size}\label{size}

Is the file big? How many records do we expect to find? (No answers
required.)

    \begin{Verbatim}[commandchars=\\\{\}]
{\color{incolor}In [{\color{incolor}5}]:} \PY{c+c1}{\PYZsh{} Run this cell to view some metadata.  No further action is needed}
        \PY{n+nb}{print}\PY{p}{(}\PY{l+s+s2}{\PYZdq{}}\PY{l+s+s2}{Size:}\PY{l+s+s2}{\PYZdq{}}\PY{p}{,} \PY{p}{(}\PY{n}{data\PYZus{}dir}\PY{o}{/}\PY{l+s+s2}{\PYZdq{}}\PY{l+s+s2}{bikeshare.txt}\PY{l+s+s2}{\PYZdq{}}\PY{p}{)}\PY{o}{.}\PY{n}{stat}\PY{p}{(}\PY{p}{)}\PY{o}{.}\PY{n}{st\PYZus{}size}\PY{p}{,} \PY{l+s+s2}{\PYZdq{}}\PY{l+s+s2}{bytes}\PY{l+s+s2}{\PYZdq{}}\PY{p}{)}
        \PY{n+nb}{print}\PY{p}{(}\PY{l+s+s2}{\PYZdq{}}\PY{l+s+s2}{Line Count:}\PY{l+s+s2}{\PYZdq{}}\PY{p}{,} \PY{n}{ds100\PYZus{}utils}\PY{o}{.}\PY{n}{line\PYZus{}count}\PY{p}{(}\PY{n}{data\PYZus{}dir}\PY{o}{/}\PY{l+s+s2}{\PYZdq{}}\PY{l+s+s2}{bikeshare.txt}\PY{l+s+s2}{\PYZdq{}}\PY{p}{)}\PY{p}{,} \PY{l+s+s2}{\PYZdq{}}\PY{l+s+s2}{lines}\PY{l+s+s2}{\PYZdq{}}\PY{p}{)}
\end{Verbatim}


    \begin{Verbatim}[commandchars=\\\{\}]
Size: 1156736 bytes
Line Count: 17380 lines

    \end{Verbatim}

    \subsubsection{Loading the data}\label{loading-the-data}

The following code loads the data into a Pandas DataFrame.

    \begin{Verbatim}[commandchars=\\\{\}]
{\color{incolor}In [{\color{incolor}6}]:} \PY{c+c1}{\PYZsh{} Run this cell to load the data.  No further action is needed}
        \PY{n}{bike} \PY{o}{=} \PY{n}{pd}\PY{o}{.}\PY{n}{read\PYZus{}csv}\PY{p}{(}\PY{n}{data\PYZus{}dir}\PY{o}{/}\PY{l+s+s1}{\PYZsq{}}\PY{l+s+s1}{bikeshare.txt}\PY{l+s+s1}{\PYZsq{}}\PY{p}{)}
        \PY{n}{bike}\PY{o}{.}\PY{n}{head}\PY{p}{(}\PY{p}{)}
\end{Verbatim}


\begin{Verbatim}[commandchars=\\\{\}]
{\color{outcolor}Out[{\color{outcolor}6}]:}    instant      dteday  season  yr  mnth  hr  holiday  weekday  workingday  \textbackslash{}
        0        1  2011-01-01       1   0     1   0        0        6           0   
        1        2  2011-01-01       1   0     1   1        0        6           0   
        2        3  2011-01-01       1   0     1   2        0        6           0   
        3        4  2011-01-01       1   0     1   3        0        6           0   
        4        5  2011-01-01       1   0     1   4        0        6           0   
        
           weathersit  temp   atemp   hum  windspeed  casual  registered  cnt  
        0           1  0.24  0.2879  0.81        0.0       3          13   16  
        1           1  0.22  0.2727  0.80        0.0       8          32   40  
        2           1  0.22  0.2727  0.80        0.0       5          27   32  
        3           1  0.24  0.2879  0.75        0.0       3          10   13  
        4           1  0.24  0.2879  0.75        0.0       0           1    1  
\end{Verbatim}
            
    Below, we show the shape of the file. You should see that the size of
the dataframe matches the number of lines in the file, minus the header
row.

    \begin{Verbatim}[commandchars=\\\{\}]
{\color{incolor}In [{\color{incolor}7}]:} \PY{n}{bike}\PY{o}{.}\PY{n}{shape}
\end{Verbatim}


\begin{Verbatim}[commandchars=\\\{\}]
{\color{outcolor}Out[{\color{outcolor}7}]:} (17379, 17)
\end{Verbatim}
            
    \begin{center}\rule{0.5\linewidth}{\linethickness}\end{center}

\subsection{1: Data Preparation}\label{data-preparation}

A few of the variables that are numeric/integer actually encode
categorical data. These include \texttt{holiday}, \texttt{weekday},
\texttt{workingday}, and \texttt{weathersit}. In the following problem,
we will convert these four variables to strings specifying the
categories. In particular, use 3-letter labels (\texttt{Sun},
\texttt{Mon}, \texttt{Tue}, \texttt{Wed}, \texttt{Thu}, \texttt{Fri},
and \texttt{Sat}) for \texttt{weekday}. You may simply use
\texttt{yes}/\texttt{no} for \texttt{holiday} and \texttt{workingday}.

In this exercise we will \emph{mutate} the data frame,
\textbf{overwriting the corresponding variables in the data frame.}
However, our notebook will effectively document this in-place data
transformation for future readers. Make sure to leave the underlying
datafile \texttt{bikeshare.txt} unmodified.

    \subsubsection{Question 1}\label{question-1}

\paragraph{\texorpdfstring{Question 1a (Decoding \texttt{weekday},
\texttt{workingday}, and
\texttt{weathersit})}{Question 1a (Decoding weekday, workingday, and weathersit)}}\label{question-1a-decoding-weekday-workingday-and-weathersit}

Decode the \texttt{holiday}, \texttt{weekday}, \texttt{workingday}, and
\texttt{weathersit} fields:

\begin{enumerate}
\def\labelenumi{\arabic{enumi}.}
\tightlist
\item
  \texttt{holiday}: Convert to \texttt{yes} and \texttt{no}. Hint: There
  are fewer holidays...
\item
  \texttt{weekday}: It turns out that Monday is the day with the most
  holidays. Mutate the
  \texttt{\textquotesingle{}weekday\textquotesingle{}} column to use the
  3-letter label (\texttt{\textquotesingle{}Sun\textquotesingle{}},
  \texttt{\textquotesingle{}Mon\textquotesingle{}},
  \texttt{\textquotesingle{}Tue\textquotesingle{}},
  \texttt{\textquotesingle{}Wed\textquotesingle{}},
  \texttt{\textquotesingle{}Thu\textquotesingle{}},
  \texttt{\textquotesingle{}Fri\textquotesingle{}}, and
  \texttt{\textquotesingle{}Sat\textquotesingle{}} ...) instead of its
  current numerical values. Assume \texttt{0} corresponds to
  \texttt{Sun}, \texttt{1} to \texttt{Mon} and so on.
\item
  \texttt{workingday}: Convert to \texttt{yes} and \texttt{no}.
\item
  \texttt{weathersit}: You should replace each value with one of
  \texttt{Clear}, \texttt{Mist}, \texttt{Light}, or \texttt{Heavy}.
\end{enumerate}

Note if you want to revert the changes run the cell that reloads the
csv.

\textbf{Hint:} One approach is to use the replace method of the pandas
DataFrame class. We haven't discussed how to do this so you'll need to
look at the documentation. The most concise way is with the approach
described in the documentation as "nested-dictonaries", though there are
many possible solutions.

    \begin{Verbatim}[commandchars=\\\{\}]
{\color{incolor}In [{\color{incolor}8}]:} \PY{c+c1}{\PYZsh{} Modify holiday weekday, workingday, and weathersit here}
        \PY{c+c1}{\PYZsh{} Hint: one strategy involves df.replace(...)}
        
        \PY{n}{nested\PYZus{}dic} \PY{o}{=} \PY{p}{\PYZob{}}\PY{l+s+s1}{\PYZsq{}}\PY{l+s+s1}{weekday}\PY{l+s+s1}{\PYZsq{}}\PY{p}{:}\PY{p}{\PYZob{}}\PY{l+m+mi}{0}\PY{p}{:}\PY{l+s+s1}{\PYZsq{}}\PY{l+s+s1}{Sun}\PY{l+s+s1}{\PYZsq{}}\PY{p}{,} \PY{l+m+mi}{1}\PY{p}{:}\PY{l+s+s1}{\PYZsq{}}\PY{l+s+s1}{Mon}\PY{l+s+s1}{\PYZsq{}}\PY{p}{,} \PY{l+m+mi}{2}\PY{p}{:}\PY{l+s+s1}{\PYZsq{}}\PY{l+s+s1}{Tue}\PY{l+s+s1}{\PYZsq{}}\PY{p}{,} \PY{l+m+mi}{3}\PY{p}{:}\PY{l+s+s1}{\PYZsq{}}\PY{l+s+s1}{Wed}\PY{l+s+s1}{\PYZsq{}}\PY{p}{,} \PY{l+m+mi}{4}\PY{p}{:}\PY{l+s+s1}{\PYZsq{}}\PY{l+s+s1}{Thu}\PY{l+s+s1}{\PYZsq{}}\PY{p}{,} \PY{l+m+mi}{5}\PY{p}{:}\PY{l+s+s1}{\PYZsq{}}\PY{l+s+s1}{Fri}\PY{l+s+s1}{\PYZsq{}}\PY{p}{,} \PY{l+m+mi}{6}\PY{p}{:}\PY{l+s+s1}{\PYZsq{}}\PY{l+s+s1}{Sat}\PY{l+s+s1}{\PYZsq{}}\PY{p}{\PYZcb{}}\PY{p}{,} 
                      \PY{l+s+s1}{\PYZsq{}}\PY{l+s+s1}{holiday}\PY{l+s+s1}{\PYZsq{}}\PY{p}{:}\PY{p}{\PYZob{}}\PY{l+m+mi}{0}\PY{p}{:}\PY{l+s+s1}{\PYZsq{}}\PY{l+s+s1}{no}\PY{l+s+s1}{\PYZsq{}}\PY{p}{,} \PY{l+m+mi}{1}\PY{p}{:}\PY{l+s+s1}{\PYZsq{}}\PY{l+s+s1}{yes}\PY{l+s+s1}{\PYZsq{}}\PY{p}{\PYZcb{}}\PY{p}{,} 
                      \PY{l+s+s1}{\PYZsq{}}\PY{l+s+s1}{workingday}\PY{l+s+s1}{\PYZsq{}}\PY{p}{:}\PY{p}{\PYZob{}}\PY{l+m+mi}{0}\PY{p}{:}\PY{l+s+s1}{\PYZsq{}}\PY{l+s+s1}{no}\PY{l+s+s1}{\PYZsq{}}\PY{p}{,} \PY{l+m+mi}{1}\PY{p}{:}\PY{l+s+s1}{\PYZsq{}}\PY{l+s+s1}{yes}\PY{l+s+s1}{\PYZsq{}}\PY{p}{\PYZcb{}}\PY{p}{,} 
                      \PY{l+s+s1}{\PYZsq{}}\PY{l+s+s1}{weathersit}\PY{l+s+s1}{\PYZsq{}}\PY{p}{:}\PY{p}{\PYZob{}}\PY{l+m+mi}{1}\PY{p}{:}\PY{l+s+s1}{\PYZsq{}}\PY{l+s+s1}{Clear}\PY{l+s+s1}{\PYZsq{}}\PY{p}{,} \PY{l+m+mi}{2}\PY{p}{:}\PY{l+s+s1}{\PYZsq{}}\PY{l+s+s1}{Mist}\PY{l+s+s1}{\PYZsq{}}\PY{p}{,} \PY{l+m+mi}{3}\PY{p}{:}\PY{l+s+s1}{\PYZsq{}}\PY{l+s+s1}{Light}\PY{l+s+s1}{\PYZsq{}}\PY{p}{,} \PY{l+m+mi}{4}\PY{p}{:}\PY{l+s+s1}{\PYZsq{}}\PY{l+s+s1}{Heavy}\PY{l+s+s1}{\PYZsq{}}\PY{p}{\PYZcb{}}\PY{p}{\PYZcb{}}
        \PY{n}{bike} \PY{o}{=} \PY{n}{bike}\PY{o}{.}\PY{n}{replace}\PY{p}{(}\PY{n}{nested\PYZus{}dic}\PY{p}{)}
        
        
        \PY{c+c1}{\PYZsh{} raise NotImplementedError()}
\end{Verbatim}


    \begin{Verbatim}[commandchars=\\\{\}]
{\color{incolor}In [{\color{incolor}9}]:} \PY{k}{assert} \PY{n+nb}{isinstance}\PY{p}{(}\PY{n}{bike}\PY{p}{,} \PY{n}{pd}\PY{o}{.}\PY{n}{DataFrame}\PY{p}{)}
        \PY{k}{assert} \PY{n}{bike}\PY{p}{[}\PY{l+s+s1}{\PYZsq{}}\PY{l+s+s1}{holiday}\PY{l+s+s1}{\PYZsq{}}\PY{p}{]}\PY{o}{.}\PY{n}{dtype} \PY{o}{==} \PY{n}{np}\PY{o}{.}\PY{n}{dtype}\PY{p}{(}\PY{l+s+s1}{\PYZsq{}}\PY{l+s+s1}{O}\PY{l+s+s1}{\PYZsq{}}\PY{p}{)}
        \PY{k}{assert} \PY{n+nb}{list}\PY{p}{(}\PY{n}{bike}\PY{p}{[}\PY{l+s+s1}{\PYZsq{}}\PY{l+s+s1}{holiday}\PY{l+s+s1}{\PYZsq{}}\PY{p}{]}\PY{o}{.}\PY{n}{iloc}\PY{p}{[}\PY{l+m+mi}{370}\PY{p}{:}\PY{l+m+mi}{375}\PY{p}{]}\PY{p}{)} \PY{o}{==} \PY{p}{[}\PY{l+s+s1}{\PYZsq{}}\PY{l+s+s1}{no}\PY{l+s+s1}{\PYZsq{}}\PY{p}{,} \PY{l+s+s1}{\PYZsq{}}\PY{l+s+s1}{no}\PY{l+s+s1}{\PYZsq{}}\PY{p}{,} \PY{l+s+s1}{\PYZsq{}}\PY{l+s+s1}{yes}\PY{l+s+s1}{\PYZsq{}}\PY{p}{,} \PY{l+s+s1}{\PYZsq{}}\PY{l+s+s1}{yes}\PY{l+s+s1}{\PYZsq{}}\PY{p}{,} \PY{l+s+s1}{\PYZsq{}}\PY{l+s+s1}{yes}\PY{l+s+s1}{\PYZsq{}}\PY{p}{]}
        \PY{k}{assert} \PY{n}{bike}\PY{p}{[}\PY{l+s+s1}{\PYZsq{}}\PY{l+s+s1}{weekday}\PY{l+s+s1}{\PYZsq{}}\PY{p}{]}\PY{o}{.}\PY{n}{dtype} \PY{o}{==} \PY{n}{np}\PY{o}{.}\PY{n}{dtype}\PY{p}{(}\PY{l+s+s1}{\PYZsq{}}\PY{l+s+s1}{O}\PY{l+s+s1}{\PYZsq{}}\PY{p}{)}
        \PY{k}{assert} \PY{n}{bike}\PY{p}{[}\PY{l+s+s1}{\PYZsq{}}\PY{l+s+s1}{workingday}\PY{l+s+s1}{\PYZsq{}}\PY{p}{]}\PY{o}{.}\PY{n}{dtype} \PY{o}{==} \PY{n}{np}\PY{o}{.}\PY{n}{dtype}\PY{p}{(}\PY{l+s+s1}{\PYZsq{}}\PY{l+s+s1}{O}\PY{l+s+s1}{\PYZsq{}}\PY{p}{)}
        \PY{k}{assert} \PY{n}{bike}\PY{p}{[}\PY{l+s+s1}{\PYZsq{}}\PY{l+s+s1}{weathersit}\PY{l+s+s1}{\PYZsq{}}\PY{p}{]}\PY{o}{.}\PY{n}{dtype} \PY{o}{==} \PY{n}{np}\PY{o}{.}\PY{n}{dtype}\PY{p}{(}\PY{l+s+s1}{\PYZsq{}}\PY{l+s+s1}{O}\PY{l+s+s1}{\PYZsq{}}\PY{p}{)}
\end{Verbatim}


    \paragraph{Question 1b (Holidays)}\label{question-1b-holidays}

How many entries in the data correspond to holidays? Set the variable
\texttt{num\_holidays} to this value.

    \begin{Verbatim}[commandchars=\\\{\}]
{\color{incolor}In [{\color{incolor}10}]:} \PY{n}{num\PYZus{}holidays} \PY{o}{=} \PY{n+nb}{len}\PY{p}{(}\PY{n}{bike}\PY{p}{[}\PY{n}{bike}\PY{p}{[}\PY{l+s+s1}{\PYZsq{}}\PY{l+s+s1}{holiday}\PY{l+s+s1}{\PYZsq{}}\PY{p}{]}\PY{o}{==}\PY{l+s+s1}{\PYZsq{}}\PY{l+s+s1}{yes}\PY{l+s+s1}{\PYZsq{}}\PY{p}{]}\PY{p}{)}
         \PY{c+c1}{\PYZsh{} YOUR CODE HERE}
         \PY{c+c1}{\PYZsh{} raise NotImplementedError()}
\end{Verbatim}


    \begin{Verbatim}[commandchars=\\\{\}]
{\color{incolor}In [{\color{incolor}11}]:} \PY{k}{assert} \PY{l+m+mi}{400} \PY{o}{\PYZlt{}}\PY{o}{=} \PY{n}{num\PYZus{}holidays} \PY{o}{\PYZlt{}}\PY{o}{=} \PY{l+m+mi}{550}
\end{Verbatim}


    \paragraph{Question 1c (Computing Daily Total
Counts)}\label{question-1c-computing-daily-total-counts}

The granularity of this data is at the hourly level. However, for some
of the analysis we will also want to compute daily statistics. In
particular, in the next few questions we will be analyzing the daily
number of registered and unregistered users.

Construct a data frame with the following columns: * \texttt{casual}:
total number of casual riders for each day * \texttt{registered}: total
number of registered riders for each day * \texttt{workingday}: whether
that day is a working day or not (\texttt{yes} or \texttt{no})

\textbf{Hint}: \texttt{groupby} and \texttt{agg}. For the \texttt{agg}
method, please check the
\href{https://pandas.pydata.org/pandas-docs/stable/generated/pandas.core.groupby.DataFrameGroupBy.agg.html}{documentation}
for examples on applying different aggregations per column. If you use
the capability to do different aggregations by column, you can do this
task with a single call to \texttt{groupby} and \texttt{agg}. For the
\texttt{workingday} column we can take any of the values since we are
grouping by the day, thus the value will be the same within each group.
Take a look at the \texttt{\textquotesingle{}first\textquotesingle{}} or
\texttt{\textquotesingle{}last\textquotesingle{}} aggregation functions.

    \begin{Verbatim}[commandchars=\\\{\}]
{\color{incolor}In [{\color{incolor}12}]:} \PY{n}{daily\PYZus{}counts} \PY{o}{=} \PY{n}{bike}\PY{o}{.}\PY{n}{groupby}\PY{p}{(}\PY{n}{by}\PY{o}{=}\PY{l+s+s1}{\PYZsq{}}\PY{l+s+s1}{dteday}\PY{l+s+s1}{\PYZsq{}}\PY{p}{)}\PY{o}{.}\PY{n}{agg}\PY{p}{(}\PY{p}{(}\PY{p}{\PYZob{}}\PY{l+s+s1}{\PYZsq{}}\PY{l+s+s1}{casual}\PY{l+s+s1}{\PYZsq{}}\PY{p}{:} \PY{l+s+s1}{\PYZsq{}}\PY{l+s+s1}{sum}\PY{l+s+s1}{\PYZsq{}}\PY{p}{,} \PY{l+s+s1}{\PYZsq{}}\PY{l+s+s1}{registered}\PY{l+s+s1}{\PYZsq{}}\PY{p}{:} \PY{l+s+s1}{\PYZsq{}}\PY{l+s+s1}{sum}\PY{l+s+s1}{\PYZsq{}}\PY{p}{,} \PY{l+s+s1}{\PYZsq{}}\PY{l+s+s1}{workingday}\PY{l+s+s1}{\PYZsq{}}\PY{p}{:}\PY{l+s+s1}{\PYZsq{}}\PY{l+s+s1}{first}\PY{l+s+s1}{\PYZsq{}}\PY{p}{\PYZcb{}}\PY{p}{)}\PY{p}{)}
         
         \PY{c+c1}{\PYZsh{} YOUR CODE HERE}
         \PY{c+c1}{\PYZsh{} raise NotImplementedError()}
\end{Verbatim}


    \begin{Verbatim}[commandchars=\\\{\}]
{\color{incolor}In [{\color{incolor}13}]:} \PY{k}{assert} \PY{n}{np}\PY{o}{.}\PY{n}{round}\PY{p}{(}\PY{n}{daily\PYZus{}counts}\PY{p}{[}\PY{l+s+s1}{\PYZsq{}}\PY{l+s+s1}{casual}\PY{l+s+s1}{\PYZsq{}}\PY{p}{]}\PY{o}{.}\PY{n}{mean}\PY{p}{(}\PY{p}{)}\PY{p}{)} \PY{o}{==} \PY{l+m+mf}{848.0}
         \PY{k}{assert} \PY{n}{np}\PY{o}{.}\PY{n}{round}\PY{p}{(}\PY{n}{daily\PYZus{}counts}\PY{p}{[}\PY{l+s+s1}{\PYZsq{}}\PY{l+s+s1}{casual}\PY{l+s+s1}{\PYZsq{}}\PY{p}{]}\PY{o}{.}\PY{n}{var}\PY{p}{(}\PY{p}{)}\PY{p}{)} \PY{o}{==} \PY{l+m+mf}{471450.0}
\end{Verbatim}


    \begin{center}\rule{0.5\linewidth}{\linethickness}\end{center}

\subsection{2: Exploring the Distribution of
Riders}\label{exploring-the-distribution-of-riders}

Let's begin by comparing the distribution of the daily counts of casual
and registered riders.

    \subsubsection{Question 2}\label{question-2}

\paragraph{Question 2a}\label{question-2a}

Use the
\href{https://seaborn.pydata.org/generated/seaborn.distplot.html}{\texttt{sns.distplot}}
function to create a plot that overlays the distribution of the daily
counts of \texttt{casual} and \texttt{registered} users. The temporal
granularity of the records should be daily counts, which you should have
after completing question 1c.

Include a legend, xlabel, ylabel, and title. You may want to look at the
\href{https://seaborn.pydata.org/tutorial/distributions.html}{seaborn
plotting tutorial} if you're not sure how to add these. After creating
the plot, look at it and make sure you understand what the plot is
actually telling us, e.g on a given day, the most likely number of
registered riders we expect is \textasciitilde{}4000, but it could be
anywher from almost none to 7000.

    \begin{Verbatim}[commandchars=\\\{\}]
{\color{incolor}In [{\color{incolor}14}]:} \PY{n}{fig} \PY{o}{=} \PY{n}{sns}\PY{o}{.}\PY{n}{distplot}\PY{p}{(}\PY{n}{daily\PYZus{}counts}\PY{p}{[}\PY{l+s+s1}{\PYZsq{}}\PY{l+s+s1}{casual}\PY{l+s+s1}{\PYZsq{}}\PY{p}{]}\PY{p}{,} \PY{n}{kde}\PY{o}{=}\PY{k+kc}{True}\PY{p}{,} \PY{n}{rug}\PY{o}{=}\PY{k+kc}{False}\PY{p}{,} \PY{n}{label}\PY{o}{=}\PY{l+s+s2}{\PYZdq{}}\PY{l+s+s2}{casual}\PY{l+s+s2}{\PYZdq{}}\PY{p}{)}\PY{p}{;}
         \PY{n}{fig} \PY{o}{=} \PY{n}{sns}\PY{o}{.}\PY{n}{distplot}\PY{p}{(}\PY{n}{daily\PYZus{}counts}\PY{p}{[}\PY{l+s+s1}{\PYZsq{}}\PY{l+s+s1}{registered}\PY{l+s+s1}{\PYZsq{}}\PY{p}{]}\PY{p}{,} \PY{n}{kde}\PY{o}{=}\PY{k+kc}{True}\PY{p}{,} \PY{n}{rug}\PY{o}{=}\PY{k+kc}{False}\PY{p}{,} \PY{n}{label}\PY{o}{=}\PY{l+s+s2}{\PYZdq{}}\PY{l+s+s2}{registered}\PY{l+s+s2}{\PYZdq{}}\PY{p}{,} \PY{n}{color}\PY{o}{=}\PY{l+s+s1}{\PYZsq{}}\PY{l+s+s1}{g}\PY{l+s+s1}{\PYZsq{}}\PY{p}{)}\PY{p}{;}
         \PY{n}{fig}\PY{o}{.}\PY{n}{figure}\PY{o}{.}\PY{n}{suptitle}\PY{p}{(}\PY{l+s+s2}{\PYZdq{}}\PY{l+s+s2}{Distribution Comparison of Casual vs Registered Riders}\PY{l+s+s2}{\PYZdq{}}\PY{p}{,} \PY{n}{fontsize} \PY{o}{=} \PY{l+m+mi}{12}\PY{p}{)}
         
         
         
         \PY{n}{plt}\PY{o}{.}\PY{n}{ylabel}\PY{p}{(}\PY{l+s+s1}{\PYZsq{}}\PY{l+s+s1}{Density}\PY{l+s+s1}{\PYZsq{}}\PY{p}{,} \PY{n}{fontsize}\PY{o}{=}\PY{l+m+mi}{14}\PY{p}{)}
         \PY{n}{plt}\PY{o}{.}\PY{n}{xlabel}\PY{p}{(}\PY{l+s+s1}{\PYZsq{}}\PY{l+s+s1}{Rider Count}\PY{l+s+s1}{\PYZsq{}}\PY{p}{,} \PY{n}{fontsize}\PY{o}{=}\PY{l+m+mi}{16}\PY{p}{)}
         \PY{n}{plt}\PY{o}{.}\PY{n}{legend}\PY{p}{(}\PY{p}{)}
\end{Verbatim}


\begin{Verbatim}[commandchars=\\\{\}]
{\color{outcolor}Out[{\color{outcolor}14}]:} <matplotlib.legend.Legend at 0x7f173d689dd8>
\end{Verbatim}
            
    \begin{center}
    \adjustimage{max size={0.9\linewidth}{0.9\paperheight}}{output_28_1.png}
    \end{center}
    { \hspace*{\fill} \\}
    
    \subsubsection{Question 2b}\label{question-2b}

In the cell below, descibe the differences you notice between the
density curves for casual and registered riders. Consider concepts such
as modes, symmetry, skewness, tails, gaps and outliers. Include a
comment on the spread of the distributions.

    \begin{Verbatim}[commandchars=\\\{\}]
{\color{incolor}In [{\color{incolor}15}]:} \PY{n}{q2b} \PY{o}{=} \PY{p}{(}\PY{l+s+s2}{\PYZdq{}}\PY{l+s+s2}{The casual riders data is skewed right and has less}\PY{l+s+s2}{\PYZdq{}}
         \PY{l+s+s2}{\PYZdq{}}\PY{l+s+s2}{ variation that is why it is closer together; the tail on }\PY{l+s+s2}{\PYZdq{}}
         \PY{l+s+s2}{\PYZdq{}}\PY{l+s+s2}{the right side shows this skew. The mode of the casual }\PY{l+s+s2}{\PYZdq{}}
         \PY{l+s+s2}{\PYZdq{}}\PY{l+s+s2}{riders is less than the mode of the registered riders. }\PY{l+s+s2}{\PYZdq{}}
         \PY{l+s+s2}{\PYZdq{}}\PY{l+s+s2}{The registered rider curve is symmetric indicated by }\PY{l+s+s2}{\PYZdq{}}
         \PY{l+s+s2}{\PYZdq{}}\PY{l+s+s2}{its symmetric tails while the casual riders curve is not }\PY{l+s+s2}{\PYZdq{}}
         \PY{l+s+s2}{\PYZdq{}}\PY{l+s+s2}{symmetric. The casual riders data has outliers that extend}\PY{l+s+s2}{\PYZdq{}}
         \PY{l+s+s2}{\PYZdq{}}\PY{l+s+s2}{ to the increasing rider count which also accounts for its }\PY{l+s+s2}{\PYZdq{}}
         \PY{l+s+s2}{\PYZdq{}}\PY{l+s+s2}{right skew. The spread of the distribution of the casual }\PY{l+s+s2}{\PYZdq{}}
         \PY{l+s+s2}{\PYZdq{}}\PY{l+s+s2}{data is less so it has less variation while the spread of }\PY{l+s+s2}{\PYZdq{}}
         \PY{l+s+s2}{\PYZdq{}}\PY{l+s+s2}{the distribution of the registered riders data is wider }\PY{l+s+s2}{\PYZdq{}}
         \PY{l+s+s2}{\PYZdq{}}\PY{l+s+s2}{so it has more variation.}\PY{l+s+s2}{\PYZdq{}}\PY{p}{)}
         
         \PY{c+c1}{\PYZsh{} YOUR CODE HERE}
         \PY{c+c1}{\PYZsh{} raise NotImplementedError()}
\end{Verbatim}


    \subsubsection{Question 2c}\label{question-2c}

The density plots do not show us how the daily counts for registered and
casual riders vary together. Use
\href{https://seaborn.pydata.org/generated/seaborn.lmplot.html}{\texttt{sns.lmplot}}
to make a scatter plot to investigate the relationship between casual
and registered counts. The \texttt{lmplot} function will also try to
draw a linear regression line (just as you saw in Data 8). Color the
points in the scatterplot according to whether or not the day is working
day. There are many points in the scatter plot so make them small to
help with over plotting. Also make sure to set \texttt{fit\_reg=True} to
generate the linear regression line. You can set the \texttt{height}
parameter if you want to adjust the size of the \texttt{lmplot}. Make
sure to include a title.

\textbf{Hints:} * Checkout this helpful
\href{https://seaborn.pydata.org/tutorial/regression.html}{tutorial on
\texttt{lmplot}}.

\begin{itemize}
\tightlist
\item
  You will need to set \texttt{x}, \texttt{y}, and \texttt{hue} and the
  \texttt{scatter\_kws}.
\end{itemize}

    \begin{Verbatim}[commandchars=\\\{\}]
{\color{incolor}In [{\color{incolor}16}]:} \PY{c+c1}{\PYZsh{} Make the font size a bit bigger}
         \PY{n}{sns}\PY{o}{.}\PY{n}{set}\PY{p}{(}\PY{n}{font\PYZus{}scale}\PY{o}{=}\PY{l+m+mf}{1.5}\PY{p}{)}
         \PY{c+c1}{\PYZsh{} Make the font size a bit bigger}
         \PY{n}{sns}\PY{o}{.}\PY{n}{set}\PY{p}{(}\PY{n}{font\PYZus{}scale}\PY{o}{=}\PY{l+m+mf}{1.5}\PY{p}{)}
         
         \PY{n}{graph} \PY{o}{=} \PY{n}{sns}\PY{o}{.}\PY{n}{lmplot}\PY{p}{(}\PY{n}{x}\PY{o}{=}\PY{l+s+s1}{\PYZsq{}}\PY{l+s+s1}{casual}\PY{l+s+s1}{\PYZsq{}}\PY{p}{,} \PY{n}{y}\PY{o}{=}\PY{l+s+s1}{\PYZsq{}}\PY{l+s+s1}{registered}\PY{l+s+s1}{\PYZsq{}}\PY{p}{,} \PY{n}{data}\PY{o}{=}\PY{n}{bike}\PY{p}{,}\PY{n}{hue}\PY{o}{=}\PY{l+s+s1}{\PYZsq{}}\PY{l+s+s1}{workingday}\PY{l+s+s1}{\PYZsq{}}\PY{p}{,}\PY{n}{fit\PYZus{}reg}\PY{o}{=}\PY{k+kc}{True}\PY{p}{,} \PY{n}{size}\PY{o}{=}\PY{l+m+mi}{10}\PY{p}{,} \PY{n}{scatter\PYZus{}kws}\PY{o}{=}\PY{p}{\PYZob{}}\PY{l+s+s2}{\PYZdq{}}\PY{l+s+s2}{s}\PY{l+s+s2}{\PYZdq{}}\PY{p}{:}\PY{l+m+mi}{9}\PY{p}{\PYZcb{}}\PY{p}{)}
         \PY{c+c1}{\PYZsh{} graph.suptitle(\PYZdq{}Comparison of Casual vs Registered Riders on Working and Non\PYZhy{}Working Days\PYZdq{}, fontsize = 12)}
         \PY{n}{ax} \PY{o}{=} \PY{n}{plt}\PY{o}{.}\PY{n}{gca}\PY{p}{(}\PY{p}{)}
         \PY{n}{ax}\PY{o}{.}\PY{n}{set\PYZus{}title}\PY{p}{(}\PY{l+s+s2}{\PYZdq{}}\PY{l+s+s2}{Comparison of Casual vs Registered Riders on Working and Non\PYZhy{}Working Days}\PY{l+s+s2}{\PYZdq{}}\PY{p}{)}
\end{Verbatim}


    \begin{Verbatim}[commandchars=\\\{\}]
/srv/conda/envs/data100/lib/python3.6/site-packages/seaborn/regression.py:546: UserWarning: The `size` paramter has been renamed to `height`; please update your code.
  warnings.warn(msg, UserWarning)

    \end{Verbatim}

\begin{Verbatim}[commandchars=\\\{\}]
{\color{outcolor}Out[{\color{outcolor}16}]:} Text(0.5,1,'Comparison of Casual vs Registered Riders on Working and Non-Working Days')
\end{Verbatim}
            
    \begin{center}
    \adjustimage{max size={0.9\linewidth}{0.9\paperheight}}{output_32_2.png}
    \end{center}
    { \hspace*{\fill} \\}
    
    \subsubsection{Question 2d}\label{question-2d}

What does this scatterplot seem to reveal about the relationship (if
any) between casual and registered riders and whether or not the day is
on the weekend?

Why might we be concerned with overplotting in examining this
relationship? By "overplotting", we're referring to the term used in
chapter 6.5 of the
\href{http://www.textbook.ds100.org/ch/06/viz_principles_2.html}{textbook}.

    \begin{Verbatim}[commandchars=\\\{\}]
{\color{incolor}In [{\color{incolor}17}]:} \PY{n}{q2d} \PY{o}{=} \PY{p}{(}\PY{l+s+s2}{\PYZdq{}}\PY{l+s+s2}{This plot reveals that casual and registered riders }\PY{l+s+s2}{\PYZdq{}}
         \PY{l+s+s2}{\PYZdq{}}\PY{l+s+s2}{have the most riders on different parts of the week }\PY{l+s+s2}{\PYZdq{}}
         \PY{l+s+s2}{\PYZdq{}}\PY{l+s+s2}{(weekend vs weekdays). When it is a workday there are more }\PY{l+s+s2}{\PYZdq{}}
         \PY{l+s+s2}{\PYZdq{}}\PY{l+s+s2}{registered riders and when it is a weekend there are more }\PY{l+s+s2}{\PYZdq{}}
         \PY{l+s+s2}{\PYZdq{}}\PY{l+s+s2}{casual riders. We are overplotting because we are looking at }\PY{l+s+s2}{\PYZdq{}}
         \PY{l+s+s2}{\PYZdq{}}\PY{l+s+s2}{too many data points so you can}\PY{l+s+s2}{\PYZsq{}}\PY{l+s+s2}{t really see what the data }\PY{l+s+s2}{\PYZdq{}}
         \PY{l+s+s2}{\PYZdq{}}\PY{l+s+s2}{is when there are lower numbers or casual and registered }\PY{l+s+s2}{\PYZdq{}}
         \PY{l+s+s2}{\PYZdq{}}\PY{l+s+s2}{riders. The data is also over lapping (workingday overlapping nonworkday)}\PY{l+s+s2}{\PYZdq{}}
         \PY{l+s+s2}{\PYZdq{}}\PY{l+s+s2}{so you can}\PY{l+s+s2}{\PYZsq{}}\PY{l+s+s2}{t see some of the points. This can be fixed with a }\PY{l+s+s2}{\PYZdq{}}
         \PY{l+s+s2}{\PYZdq{}}\PY{l+s+s2}{kernel density estimator function.}\PY{l+s+s2}{\PYZdq{}}\PY{p}{)}
         
         \PY{c+c1}{\PYZsh{} YOUR CODE HERE}
         \PY{c+c1}{\PYZsh{} raise NotImplementedError()}
\end{Verbatim}


    \begin{center}\rule{0.5\linewidth}{\linethickness}\end{center}

\subsection{3: Visualization}\label{visualization}

\subsubsection{Question 3}\label{question-3}

\paragraph{Question 3a Bivariate Kernel Density
Plot}\label{question-3a-bivariate-kernel-density-plot}

The scatter plot you made in question 2c makes clear the separation
between the work days and non-work days. However, the overplotting makes
it difficult to see the density of the joint counts. To address this
issue, let's try visualizing the data with another technique, the
bivariate kernel density estimate.

You will want to read up on the documentation for \texttt{sns.kdeplot}
which can be found at
https://seaborn.pydata.org/generated/seaborn.kdeplot.html

The result we wish to achieve should be a plot that looks something like
this:

You can think of this plot as an overhead countour or topographical map,
where the "high" regions are those with more data points, and "low"
regions are those with fewer data points.

    A basic kde plot of all the data is quite easy to generate. However,
this plot includes both weekend and weekday data, which isn't what we
want (see example figure above).

    \begin{Verbatim}[commandchars=\\\{\}]
{\color{incolor}In [{\color{incolor}18}]:} \PY{n}{sns}\PY{o}{.}\PY{n}{kdeplot}\PY{p}{(}\PY{n}{daily\PYZus{}counts}\PY{p}{[}\PY{l+s+s1}{\PYZsq{}}\PY{l+s+s1}{casual}\PY{l+s+s1}{\PYZsq{}}\PY{p}{]}\PY{p}{,} \PY{n}{daily\PYZus{}counts}\PY{p}{[}\PY{l+s+s1}{\PYZsq{}}\PY{l+s+s1}{registered}\PY{l+s+s1}{\PYZsq{}}\PY{p}{]}\PY{p}{)}
         \PY{n}{plt}\PY{o}{.}\PY{n}{title}\PY{p}{(}\PY{l+s+s1}{\PYZsq{}}\PY{l+s+s1}{KDE Plot Comparison of Registered vs Casual Riders}\PY{l+s+s1}{\PYZsq{}}\PY{p}{)}\PY{p}{;}
\end{Verbatim}


    \begin{center}
    \adjustimage{max size={0.9\linewidth}{0.9\paperheight}}{output_37_0.png}
    \end{center}
    { \hspace*{\fill} \\}
    
    Generating the plot with weekend and weekday separated can be
complicated so we will provide a walkthrough below, feel free to use
whatever method you wish however if you do not want to follow the
walkthrough.

\textbf{Hints:} * You can use \texttt{loc} with a boolean array and
column names at the same time, as seen in
\href{https://docs.google.com/presentation/d/1qMRtloXaccvJLfyZckn-UpuG0Prb1XtJHLPWQX2OMSo/edit\#slide=id.g4114e65f62_0_404}{lecture
2}. * You will need to call kdeplot twice. * Check out this
\href{http://financeandpython.com/SeabornDataVisualization/8/3.html}{tutorial}
to see an example of how to set colors for each dataset and how to
create a legend. The legend part uses some weird matplotlib syntax that
we haven't learned! You'll probably find creating the legend annoying,
but it's a good exercise to learn how to use examples to get the look
you want. * You will want to set the \texttt{cmap} parameter of
\texttt{kdeplot} to \texttt{"Reds"} and \texttt{"Blues"} (or whatever
two contrasting colors you'd like).

After you get your plot working, experiment by setting
\texttt{shade=True} in \texttt{kdeplot} to see the difference between
the shaded and unshaded version. Please submit your work with
\texttt{shade=False}.

    \begin{Verbatim}[commandchars=\\\{\}]
{\color{incolor}In [{\color{incolor}19}]:} \PY{k+kn}{import} \PY{n+nn}{matplotlib}\PY{n+nn}{.}\PY{n+nn}{patches} \PY{k}{as} \PY{n+nn}{mpatches}  \PY{c+c1}{\PYZsh{} see the tutorial for how we use mpatches to generate this figure!}
         
         \PY{c+c1}{\PYZsh{} Set \PYZsq{}is\PYZus{}workingday\PYZsq{} to a boolean array that is true for all working\PYZus{}days}
         \PY{n}{plt}\PY{o}{.}\PY{n}{figure}\PY{p}{(}\PY{n}{figsize}\PY{o}{=}\PY{p}{(}\PY{l+m+mi}{10}\PY{p}{,} \PY{l+m+mi}{7}\PY{p}{)}\PY{p}{)}
         \PY{n}{is\PYZus{}workingday} \PY{o}{=} \PY{n}{daily\PYZus{}counts}\PY{p}{[}\PY{n}{daily\PYZus{}counts}\PY{p}{[}\PY{l+s+s1}{\PYZsq{}}\PY{l+s+s1}{workingday}\PY{l+s+s1}{\PYZsq{}}\PY{p}{]} \PY{o}{==} \PY{l+s+s1}{\PYZsq{}}\PY{l+s+s1}{yes}\PY{l+s+s1}{\PYZsq{}}\PY{p}{]}
         \PY{n}{not\PYZus{}workingday} \PY{o}{=} \PY{n}{daily\PYZus{}counts}\PY{p}{[}\PY{n}{daily\PYZus{}counts}\PY{p}{[}\PY{l+s+s1}{\PYZsq{}}\PY{l+s+s1}{workingday}\PY{l+s+s1}{\PYZsq{}}\PY{p}{]} \PY{o}{==} \PY{l+s+s1}{\PYZsq{}}\PY{l+s+s1}{no}\PY{l+s+s1}{\PYZsq{}}\PY{p}{]}
         
         \PY{c+c1}{\PYZsh{} Bivariate KDEs require two data inputs. }
         \PY{c+c1}{\PYZsh{} In this case, we will need the daily counts for casual and registered riders on weekdays}
         \PY{c+c1}{\PYZsh{} Hint: use loc and is\PYZus{}workingday to splice out the relevant rows and column (casual/registered). }
         \PY{n}{casual\PYZus{}weekday} \PY{o}{=} \PY{n}{is\PYZus{}workingday}\PY{o}{.}\PY{n}{loc}\PY{p}{[}\PY{p}{:}\PY{p}{]}\PY{p}{[}\PY{l+s+s1}{\PYZsq{}}\PY{l+s+s1}{casual}\PY{l+s+s1}{\PYZsq{}}\PY{p}{]}
         \PY{n}{registered\PYZus{}weekday} \PY{o}{=} \PY{n}{is\PYZus{}workingday}\PY{o}{.}\PY{n}{loc}\PY{p}{[}\PY{p}{:}\PY{p}{]}\PY{p}{[}\PY{l+s+s1}{\PYZsq{}}\PY{l+s+s1}{registered}\PY{l+s+s1}{\PYZsq{}}\PY{p}{]}
         
         \PY{n}{sns}\PY{o}{.}\PY{n}{kdeplot}\PY{p}{(}\PY{n}{casual\PYZus{}weekday}\PY{p}{,} \PY{n}{registered\PYZus{}weekday}\PY{p}{,} \PY{n}{shade}\PY{o}{=}\PY{k+kc}{False}\PY{p}{,} \PY{n}{cmap}\PY{o}{=}\PY{l+s+s1}{\PYZsq{}}\PY{l+s+s1}{Reds}\PY{l+s+s1}{\PYZsq{}}\PY{p}{)}
         
         \PY{c+c1}{\PYZsh{} Use sns.kdeplot on the two variables above to plot the bivariate KDE for weekday rides}
         
         \PY{c+c1}{\PYZsh{} Repeat the same steps above but for rows corresponding to non\PYZhy{}workingdays}
         \PY{n}{casual\PYZus{}weekend} \PY{o}{=} \PY{n}{not\PYZus{}workingday}\PY{o}{.}\PY{n}{loc}\PY{p}{[}\PY{p}{:}\PY{p}{]}\PY{p}{[}\PY{l+s+s1}{\PYZsq{}}\PY{l+s+s1}{casual}\PY{l+s+s1}{\PYZsq{}}\PY{p}{]}
         \PY{n}{registered\PYZus{}weekend} \PY{o}{=} \PY{n}{not\PYZus{}workingday}\PY{o}{.}\PY{n}{loc}\PY{p}{[}\PY{p}{:}\PY{p}{]}\PY{p}{[}\PY{l+s+s1}{\PYZsq{}}\PY{l+s+s1}{registered}\PY{l+s+s1}{\PYZsq{}}\PY{p}{]}
         
         \PY{n}{sns}\PY{o}{.}\PY{n}{kdeplot}\PY{p}{(}\PY{n}{casual\PYZus{}weekend}\PY{p}{,} \PY{n}{registered\PYZus{}weekend}\PY{p}{,} \PY{n}{shade}\PY{o}{=}\PY{k+kc}{False}\PY{p}{,} \PY{n}{cmap}\PY{o}{=}\PY{l+s+s1}{\PYZsq{}}\PY{l+s+s1}{Blues}\PY{l+s+s1}{\PYZsq{}}\PY{p}{)}
         \PY{c+c1}{\PYZsh{} Use sns.kdeplot on the two variables above to plot the bivariate KDE for weekday rides}
         
         \PY{n}{plt}\PY{o}{.}\PY{n}{title}\PY{p}{(}\PY{l+s+s1}{\PYZsq{}}\PY{l+s+s1}{KDE Plot Comparison of Registered vs Casual Riders}\PY{l+s+s1}{\PYZsq{}}\PY{p}{)}\PY{p}{;}
         \PY{n}{r} \PY{o}{=} \PY{n}{sns}\PY{o}{.}\PY{n}{color\PYZus{}palette}\PY{p}{(}\PY{l+s+s2}{\PYZdq{}}\PY{l+s+s2}{Reds}\PY{l+s+s2}{\PYZdq{}}\PY{p}{)}\PY{p}{[}\PY{l+m+mi}{2}\PY{p}{]}
         \PY{n}{b} \PY{o}{=} \PY{n}{sns}\PY{o}{.}\PY{n}{color\PYZus{}palette}\PY{p}{(}\PY{l+s+s2}{\PYZdq{}}\PY{l+s+s2}{Blues}\PY{l+s+s2}{\PYZdq{}}\PY{p}{)}\PY{p}{[}\PY{l+m+mi}{2}\PY{p}{]}
         
         
         \PY{n}{red\PYZus{}patch} \PY{o}{=} \PY{n}{mpatches}\PY{o}{.}\PY{n}{Patch}\PY{p}{(}\PY{n}{color}\PY{o}{=}\PY{n}{r}\PY{p}{,} \PY{n}{label}\PY{o}{=}\PY{l+s+s1}{\PYZsq{}}\PY{l+s+s1}{Workday}\PY{l+s+s1}{\PYZsq{}}\PY{p}{)}
         \PY{n}{blue\PYZus{}patch} \PY{o}{=} \PY{n}{mpatches}\PY{o}{.}\PY{n}{Patch}\PY{p}{(}\PY{n}{color}\PY{o}{=}\PY{n}{b}\PY{p}{,} \PY{n}{label}\PY{o}{=}\PY{l+s+s1}{\PYZsq{}}\PY{l+s+s1}{Non\PYZhy{}Workday}\PY{l+s+s1}{\PYZsq{}}\PY{p}{)}
         
         
         \PY{n}{plt}\PY{o}{.}\PY{n}{legend}\PY{p}{(}\PY{n}{handles}\PY{o}{=}\PY{p}{[}\PY{n}{red\PYZus{}patch}\PY{p}{,}\PY{n}{blue\PYZus{}patch}\PY{p}{]}\PY{p}{)}
         
         \PY{n}{plt}\PY{o}{.}\PY{n}{show}\PY{p}{(}\PY{p}{)}
         \PY{c+c1}{\PYZsh{} YOUR CODE HERE}
         \PY{c+c1}{\PYZsh{} raise NotImplementedError()}
\end{Verbatim}


    \begin{center}
    \adjustimage{max size={0.9\linewidth}{0.9\paperheight}}{output_39_0.png}
    \end{center}
    { \hspace*{\fill} \\}
    
    \paragraph{Question 3b}\label{question-3b}

What does the contour plot suggest about the relationship between casual
and registered riders for work days? For non-work days? Do you feel like
it's easier to see the relationship on this countour plot, or on the
plot you created for 2c? Why?

    \begin{Verbatim}[commandchars=\\\{\}]
{\color{incolor}In [{\color{incolor}20}]:} \PY{n}{q3b} \PY{o}{=} \PY{p}{(}\PY{l+s+s2}{\PYZdq{}}\PY{l+s+s2}{We can now see that as the number of registered riders increases}\PY{l+s+s2}{\PYZdq{}}
         \PY{l+s+s2}{\PYZdq{}}\PY{l+s+s2}{ for workdays the number of casual riders increases much more slowly. }\PY{l+s+s2}{\PYZdq{}}
         \PY{l+s+s2}{\PYZdq{}}\PY{l+s+s2}{For non\PYZhy{}workdays as the number of casual riders increases the number of }\PY{l+s+s2}{\PYZdq{}}
         \PY{l+s+s2}{\PYZdq{}}\PY{l+s+s2}{registered riders increases much more slowly. It}\PY{l+s+s2}{\PYZsq{}}\PY{l+s+s2}{s definitely easier to }\PY{l+s+s2}{\PYZdq{}}
         \PY{l+s+s2}{\PYZdq{}}\PY{l+s+s2}{see the relationship with this plot because we are no longer overplotting }\PY{l+s+s2}{\PYZdq{}}
         \PY{l+s+s2}{\PYZdq{}}\PY{l+s+s2}{the data and we can see overlapping data since its is a density function.}\PY{l+s+s2}{\PYZdq{}}\PY{p}{)}
         
         \PY{c+c1}{\PYZsh{} YOUR CODE HERE}
         \PY{c+c1}{\PYZsh{} raise NotImplementedError()}
\end{Verbatim}


    \paragraph{Question 3c}\label{question-3c}

As an alternative approach to visualizing the data, construct the
following set of three plots where the main plot shows the contours of
the kernel density estimate of daily counts for registered and casual
riders plotted together, and the two "margin" plots (at the top and
right of the figure) provide the univariate kernel density estimate of
each of these variables. Note that this plot makes it harder see the
linear relationships between casual and registered for the two different
conditions (weekday vs. weekend).

\textbf{Hints}: * The
\href{https://seaborn.pydata.org/tutorial/distributions.html}{seaborn
plotting tutorial} has examples that may be helpful. * Take a look at
\texttt{sns.jointplot} and its \texttt{kind} parameter. *
\texttt{set\_axis\_labels} can be used to rename axes on the contour
plot. * \texttt{plt.suptitle} from lab 1 can be handy for setting the
title where you want. * \texttt{plt.subplots\_adjust(top=0.9)} can help
if your title overlaps with your plot

    \begin{Verbatim}[commandchars=\\\{\}]
{\color{incolor}In [{\color{incolor}21}]:} \PY{n}{g} \PY{o}{=} \PY{n}{sns}\PY{o}{.}\PY{n}{jointplot}\PY{p}{(}\PY{n}{x}\PY{o}{=}\PY{l+s+s2}{\PYZdq{}}\PY{l+s+s2}{casual}\PY{l+s+s2}{\PYZdq{}}\PY{p}{,} \PY{n}{y}\PY{o}{=}\PY{l+s+s2}{\PYZdq{}}\PY{l+s+s2}{registered}\PY{l+s+s2}{\PYZdq{}}\PY{p}{,} \PY{n}{data}\PY{o}{=}\PY{n}{daily\PYZus{}counts}\PY{p}{,} \PY{n}{kind}\PY{o}{=}\PY{l+s+s2}{\PYZdq{}}\PY{l+s+s2}{kde}\PY{l+s+s2}{\PYZdq{}}\PY{p}{)}\PY{p}{;}
         \PY{n}{g}\PY{o}{.}\PY{n}{set\PYZus{}axis\PYZus{}labels}\PY{p}{(}\PY{l+s+s2}{\PYZdq{}}\PY{l+s+s2}{Daily Count Registered Riders}\PY{l+s+s2}{\PYZdq{}}\PY{p}{,} \PY{l+s+s2}{\PYZdq{}}\PY{l+s+s2}{Daily Count Casual Riders}\PY{l+s+s2}{\PYZdq{}}\PY{p}{)}
         \PY{n}{plt}\PY{o}{.}\PY{n}{suptitle}\PY{p}{(}\PY{l+s+s1}{\PYZsq{}}\PY{l+s+s1}{KDE Countours of Casual vs Registered Rider Count}\PY{l+s+s1}{\PYZsq{}}\PY{p}{)}\PY{p}{;}
         \PY{n}{plt}\PY{o}{.}\PY{n}{subplots\PYZus{}adjust}\PY{p}{(}\PY{n}{top}\PY{o}{=}\PY{o}{.}\PY{l+m+mi}{9}\PY{p}{)}\PY{p}{;}
\end{Verbatim}


    \begin{center}
    \adjustimage{max size={0.9\linewidth}{0.9\paperheight}}{output_43_0.png}
    \end{center}
    { \hspace*{\fill} \\}
    
    \begin{center}\rule{0.5\linewidth}{\linethickness}\end{center}

\subsection{4: Exploring Ride Sharing and
Time}\label{exploring-ride-sharing-and-time}

\subsubsection{Question 4}\label{question-4}

\paragraph{Question 4a}\label{question-4a}

Plot number of riders for each day in the month of June in 2011.

Make sure to add descriptive x-axis and y-axis labels and create a
legend to distinguish the line for casual riders and the line for
registered riders. The end result should look like the figure below. The
shaded region is a bootstrap confidence interval similar to what you
learned about in Data 8.

Make sure to include xlabel, ylabel, a legend, and a title.

\textbf{Hints:} * Add a new Series to the \texttt{bike} datafame
correpsonding to the day. You can do something similar to what you did
in \texttt{hw1} when you created the \texttt{postal\_code\_5} Seres. *
Make sure your day series is of type int. One way is to use the
\texttt{map} method of the Series class, i.e. \texttt{s.map(int)}. * Use
\texttt{sns.lineplot}.

    \begin{Verbatim}[commandchars=\\\{\}]
{\color{incolor}In [{\color{incolor}22}]:} \PY{n}{plt}\PY{o}{.}\PY{n}{figure}\PY{p}{(}\PY{n}{figsize}\PY{o}{=}\PY{p}{(}\PY{l+m+mi}{10}\PY{p}{,} \PY{l+m+mi}{7}\PY{p}{)}\PY{p}{)}
         \PY{n}{bike}\PY{p}{[}\PY{l+s+s1}{\PYZsq{}}\PY{l+s+s1}{day\PYZsh{}}\PY{l+s+s1}{\PYZsq{}}\PY{p}{]} \PY{o}{=} \PY{n}{bike}\PY{p}{[}\PY{l+s+s1}{\PYZsq{}}\PY{l+s+s1}{dteday}\PY{l+s+s1}{\PYZsq{}}\PY{p}{]}\PY{o}{.}\PY{n}{str}\PY{o}{.}\PY{n}{split}\PY{p}{(}\PY{l+s+s1}{\PYZsq{}}\PY{l+s+s1}{\PYZhy{}}\PY{l+s+s1}{\PYZsq{}}\PY{p}{)}\PY{o}{.}\PY{n}{str}\PY{p}{[}\PY{l+m+mi}{2}\PY{p}{]}\PY{o}{.}\PY{n}{map}\PY{p}{(}\PY{n+nb}{int}\PY{p}{)}
         \PY{n}{june11\PYZus{}bike} \PY{o}{=} \PY{n}{bike}\PY{p}{[}\PY{p}{(}\PY{n}{bike}\PY{p}{[}\PY{l+s+s1}{\PYZsq{}}\PY{l+s+s1}{mnth}\PY{l+s+s1}{\PYZsq{}}\PY{p}{]} \PY{o}{==} \PY{l+m+mi}{6}\PY{p}{)}\PY{o}{\PYZam{}}\PY{p}{(}\PY{n}{bike}\PY{p}{[}\PY{l+s+s1}{\PYZsq{}}\PY{l+s+s1}{yr}\PY{l+s+s1}{\PYZsq{}}\PY{p}{]} \PY{o}{==} \PY{l+m+mi}{0}\PY{p}{)}\PY{p}{]}
         \PY{n}{g}\PY{o}{=}\PY{n}{sns}\PY{o}{.}\PY{n}{lineplot}\PY{p}{(}\PY{n}{x}\PY{o}{=}\PY{n}{june11\PYZus{}bike}\PY{p}{[}\PY{l+s+s1}{\PYZsq{}}\PY{l+s+s1}{day\PYZsh{}}\PY{l+s+s1}{\PYZsq{}}\PY{p}{]}\PY{p}{,} \PY{n}{y}\PY{o}{=}\PY{l+s+s1}{\PYZsq{}}\PY{l+s+s1}{casual}\PY{l+s+s1}{\PYZsq{}}\PY{p}{,} \PY{n}{data}\PY{o}{=}\PY{n}{june11\PYZus{}bike}\PY{p}{,} \PY{n}{label}\PY{o}{=}\PY{l+s+s2}{\PYZdq{}}\PY{l+s+s2}{Casual}\PY{l+s+s2}{\PYZdq{}}\PY{p}{)}
         \PY{n}{g}\PY{o}{=}\PY{n}{sns}\PY{o}{.}\PY{n}{lineplot}\PY{p}{(}\PY{n}{x}\PY{o}{=}\PY{n}{june11\PYZus{}bike}\PY{p}{[}\PY{l+s+s1}{\PYZsq{}}\PY{l+s+s1}{day\PYZsh{}}\PY{l+s+s1}{\PYZsq{}}\PY{p}{]}\PY{p}{,} \PY{n}{y}\PY{o}{=}\PY{l+s+s1}{\PYZsq{}}\PY{l+s+s1}{registered}\PY{l+s+s1}{\PYZsq{}}\PY{p}{,} \PY{n}{data}\PY{o}{=}\PY{n}{june11\PYZus{}bike}\PY{p}{,} \PY{n}{label}\PY{o}{=}\PY{l+s+s2}{\PYZdq{}}\PY{l+s+s2}{Registered}\PY{l+s+s2}{\PYZdq{}}\PY{p}{)}
         \PY{n}{g}\PY{o}{.}\PY{n}{set}\PY{p}{(}\PY{n}{xlabel}\PY{o}{=}\PY{l+s+s1}{\PYZsq{}}\PY{l+s+s1}{Day of the Month}\PY{l+s+s1}{\PYZsq{}}\PY{p}{,} \PY{n}{ylabel}\PY{o}{=}\PY{l+s+s1}{\PYZsq{}}\PY{l+s+s1}{Number of Riders}\PY{l+s+s1}{\PYZsq{}}\PY{p}{)}
         \PY{n}{plt}\PY{o}{.}\PY{n}{suptitle}\PY{p}{(}\PY{l+s+s1}{\PYZsq{}}\PY{l+s+s1}{Count of Casual vs Registered Riders by Day}\PY{l+s+s1}{\PYZsq{}}\PY{p}{)}\PY{p}{;}
\end{Verbatim}


    \begin{center}
    \adjustimage{max size={0.9\linewidth}{0.9\paperheight}}{output_45_0.png}
    \end{center}
    { \hspace*{\fill} \\}
    
    \paragraph{Question 4b}\label{question-4b}

This plot has several interesting features. How do the number of casual
and registered riders compare for different days of the month? What is
an interesting trend and pattern you notice between the lines? Why do
you think the confidence interval for the registered riders is, on
average, wider than the confidence interval for casual riders?

    \begin{Verbatim}[commandchars=\\\{\}]
{\color{incolor}In [{\color{incolor}23}]:} \PY{n}{q4b} \PY{o}{=} \PY{p}{(}\PY{l+s+s2}{\PYZdq{}}\PY{l+s+s2}{The general trend of the registered vs casual riders is that there are}\PY{l+s+s2}{\PYZdq{}} 
         \PY{l+s+s2}{\PYZdq{}}\PY{l+s+s2}{significantly more registered riders than casual. It appears there is a pattern }\PY{l+s+s2}{\PYZdq{}}
         \PY{l+s+s2}{\PYZdq{}}\PY{l+s+s2}{that high number of registered riders is coupled with low number of casual riders }\PY{l+s+s2}{\PYZdq{}}
         \PY{l+s+s2}{\PYZdq{}}\PY{l+s+s2}{and vice versa for the same days of the week. The larger confidence interval on }\PY{l+s+s2}{\PYZdq{}}
         \PY{l+s+s2}{\PYZdq{}}\PY{l+s+s2}{the registered line means there is more variation in the data. This might be because }\PY{l+s+s2}{\PYZdq{}}
         \PY{l+s+s2}{\PYZdq{}}\PY{l+s+s2}{casual users are only using it during the weekends so it keeps the variation in }\PY{l+s+s2}{\PYZdq{}}
         \PY{l+s+s2}{\PYZdq{}}\PY{l+s+s2}{their data low whereas registered users could be using the bikes during the weeks }\PY{l+s+s2}{\PYZdq{}}
         \PY{l+s+s2}{\PYZdq{}}\PY{l+s+s2}{and weekends, increasing the volatility of their data.}\PY{l+s+s2}{\PYZdq{}}\PY{p}{)}
         
         
         \PY{c+c1}{\PYZsh{} YOUR CODE HERE}
         \PY{c+c1}{\PYZsh{} raise NotImplementedError()}
\end{Verbatim}


    \begin{center}\rule{0.5\linewidth}{\linethickness}\end{center}

\subsection{5: Understanding Daily
Patterns}\label{understanding-daily-patterns}

\subsubsection{Question 5}\label{question-5}

\paragraph{Question 5a}\label{question-5a}

Let's examine the behavior of riders by plotting the average number of
riders for each hour of the day over the \textbf{entire dataset} (not
just June 2011), stratified by rider type.

Your plot should look like the following:

    \begin{Verbatim}[commandchars=\\\{\}]
{\color{incolor}In [{\color{incolor}24}]:} \PY{n}{plt}\PY{o}{.}\PY{n}{figure}\PY{p}{(}\PY{n}{figsize}\PY{o}{=}\PY{p}{(}\PY{l+m+mi}{10}\PY{p}{,} \PY{l+m+mi}{7}\PY{p}{)}\PY{p}{)}
         \PY{n}{hour\PYZus{}averages} \PY{o}{=} \PY{n}{bike}\PY{p}{[}\PY{p}{[}\PY{l+s+s1}{\PYZsq{}}\PY{l+s+s1}{hr}\PY{l+s+s1}{\PYZsq{}}\PY{p}{,}\PY{l+s+s1}{\PYZsq{}}\PY{l+s+s1}{casual}\PY{l+s+s1}{\PYZsq{}}\PY{p}{,}\PY{l+s+s1}{\PYZsq{}}\PY{l+s+s1}{registered}\PY{l+s+s1}{\PYZsq{}}\PY{p}{]}\PY{p}{]}\PY{o}{.}\PY{n}{groupby}\PY{p}{(}\PY{n}{by}\PY{o}{=}\PY{l+s+s1}{\PYZsq{}}\PY{l+s+s1}{hr}\PY{l+s+s1}{\PYZsq{}}\PY{p}{)}\PY{o}{.}\PY{n}{mean}\PY{p}{(}\PY{p}{)}
         \PY{n}{g}\PY{o}{=}\PY{n}{sns}\PY{o}{.}\PY{n}{lineplot}\PY{p}{(}\PY{n}{x}\PY{o}{=}\PY{n}{hour\PYZus{}averages}\PY{o}{.}\PY{n}{index}\PY{p}{,} \PY{n}{y}\PY{o}{=}\PY{l+s+s1}{\PYZsq{}}\PY{l+s+s1}{casual}\PY{l+s+s1}{\PYZsq{}}\PY{p}{,} \PY{n}{data}\PY{o}{=}\PY{n}{hour\PYZus{}averages}\PY{p}{,} \PY{n}{label}\PY{o}{=}\PY{l+s+s2}{\PYZdq{}}\PY{l+s+s2}{Casual}\PY{l+s+s2}{\PYZdq{}}\PY{p}{)}
         \PY{n}{g}\PY{o}{=}\PY{n}{sns}\PY{o}{.}\PY{n}{lineplot}\PY{p}{(}\PY{n}{x}\PY{o}{=}\PY{n}{hour\PYZus{}averages}\PY{o}{.}\PY{n}{index}\PY{p}{,} \PY{n}{y}\PY{o}{=}\PY{l+s+s1}{\PYZsq{}}\PY{l+s+s1}{registered}\PY{l+s+s1}{\PYZsq{}}\PY{p}{,} \PY{n}{data}\PY{o}{=}\PY{n}{hour\PYZus{}averages}\PY{p}{,} \PY{n}{label}\PY{o}{=}\PY{l+s+s2}{\PYZdq{}}\PY{l+s+s2}{Registered}\PY{l+s+s2}{\PYZdq{}}\PY{p}{)}
         \PY{n}{g}\PY{o}{.}\PY{n}{set}\PY{p}{(}\PY{n}{xlabel}\PY{o}{=}\PY{l+s+s1}{\PYZsq{}}\PY{l+s+s1}{Hour of the Day}\PY{l+s+s1}{\PYZsq{}}\PY{p}{,} \PY{n}{ylabel}\PY{o}{=}\PY{l+s+s1}{\PYZsq{}}\PY{l+s+s1}{Average Count}\PY{l+s+s1}{\PYZsq{}}\PY{p}{)}
         \PY{n}{plt}\PY{o}{.}\PY{n}{suptitle}\PY{p}{(}\PY{l+s+s1}{\PYZsq{}}\PY{l+s+s1}{Average Count of Casual vs Registered by Hour}\PY{l+s+s1}{\PYZsq{}}\PY{p}{)}\PY{p}{;}
\end{Verbatim}


    \begin{center}
    \adjustimage{max size={0.9\linewidth}{0.9\paperheight}}{output_49_0.png}
    \end{center}
    { \hspace*{\fill} \\}
    
    \paragraph{Question 5b}\label{question-5b}

What can you observe from the plot? Hypothesize about the meaning of the
peaks in the registered riders' distribution.

    \begin{Verbatim}[commandchars=\\\{\}]
{\color{incolor}In [{\color{incolor}25}]:} \PY{n}{q5b} \PY{o}{=} \PY{p}{(}\PY{l+s+s2}{\PYZdq{}}\PY{l+s+s2}{Registered riders are using the bikes most often during the 5\PYZhy{}10 hour and the 15\PYZhy{}20 hour;}\PY{l+s+s2}{\PYZdq{}}
                \PY{l+s+s2}{\PYZdq{}}\PY{l+s+s2}{ the casual riders are using the bikes the most during the 10\PYZhy{}20 hour. I hypothesize that}\PY{l+s+s2}{\PYZdq{}} 
                \PY{l+s+s2}{\PYZdq{}}\PY{l+s+s2}{the registered users are using the bikes to ride to and from work. That is why there are }\PY{l+s+s2}{\PYZdq{}}
                \PY{l+s+s2}{\PYZdq{}}\PY{l+s+s2}{peaks during rush hour times when people are going to work and leaving work.}\PY{l+s+s2}{\PYZdq{}}\PY{p}{)}
         
         \PY{c+c1}{\PYZsh{} YOUR CODE HERE}
         \PY{c+c1}{\PYZsh{} raise NotImplementedError()}
\end{Verbatim}


    \begin{center}\rule{0.5\linewidth}{\linethickness}\end{center}

\subsection{6: Exploring Ride Sharing and
Weather}\label{exploring-ride-sharing-and-weather}

Now let's examine how the weather is affecting rider's behavior. First
let's look at how the proportion of casual riders changes as weather
changes.

\subsubsection{Question 6}\label{question-6}

\paragraph{Question 6a}\label{question-6a}

Create a new column \texttt{prop\_casual} in the \texttt{bike} dataframe
representing the proportion of casual riders out of all riders.

    \begin{Verbatim}[commandchars=\\\{\}]
{\color{incolor}In [{\color{incolor}26}]:} \PY{n}{bike}\PY{p}{[}\PY{l+s+s1}{\PYZsq{}}\PY{l+s+s1}{prop\PYZus{}casual}\PY{l+s+s1}{\PYZsq{}}\PY{p}{]} \PY{o}{=} \PY{n}{bike}\PY{p}{[}\PY{l+s+s1}{\PYZsq{}}\PY{l+s+s1}{casual}\PY{l+s+s1}{\PYZsq{}}\PY{p}{]}\PY{o}{/}\PY{p}{(} \PY{n}{bike}\PY{p}{[}\PY{l+s+s1}{\PYZsq{}}\PY{l+s+s1}{casual}\PY{l+s+s1}{\PYZsq{}}\PY{p}{]} \PY{o}{+}  \PY{n}{bike}\PY{p}{[}\PY{l+s+s1}{\PYZsq{}}\PY{l+s+s1}{registered}\PY{l+s+s1}{\PYZsq{}}\PY{p}{]}\PY{p}{)}
         
         \PY{c+c1}{\PYZsh{} YOUR CODE HERE}
         \PY{c+c1}{\PYZsh{} raise NotImplementedError()}
\end{Verbatim}


    \begin{Verbatim}[commandchars=\\\{\}]
{\color{incolor}In [{\color{incolor}27}]:} \PY{k}{assert} \PY{n+nb}{int}\PY{p}{(}\PY{n}{bike}\PY{p}{[}\PY{l+s+s2}{\PYZdq{}}\PY{l+s+s2}{prop\PYZus{}casual}\PY{l+s+s2}{\PYZdq{}}\PY{p}{]}\PY{o}{.}\PY{n}{sum}\PY{p}{(}\PY{p}{)}\PY{p}{)} \PY{o}{==} \PY{l+m+mi}{2991}
\end{Verbatim}


    \paragraph{Question 6b}\label{question-6b}

In order to examine the relationship between proportion of casual riders
and temperature, we can create a scatterplot using
\texttt{sns.scatterplot}. We can even use color/hue to encode the
information about day of week. Run the cell below, and you'll see we end
up with a big mess that is impossible to interpret.

    \begin{Verbatim}[commandchars=\\\{\}]
{\color{incolor}In [{\color{incolor}28}]:} \PY{n}{plt}\PY{o}{.}\PY{n}{figure}\PY{p}{(}\PY{n}{figsize}\PY{o}{=}\PY{p}{(}\PY{l+m+mi}{10}\PY{p}{,} \PY{l+m+mi}{7}\PY{p}{)}\PY{p}{)}
         \PY{n}{sns}\PY{o}{.}\PY{n}{scatterplot}\PY{p}{(}\PY{n}{data}\PY{o}{=}\PY{n}{bike}\PY{p}{,} \PY{n}{x}\PY{o}{=}\PY{l+s+s2}{\PYZdq{}}\PY{l+s+s2}{temp}\PY{l+s+s2}{\PYZdq{}}\PY{p}{,} \PY{n}{y}\PY{o}{=}\PY{l+s+s2}{\PYZdq{}}\PY{l+s+s2}{prop\PYZus{}casual}\PY{l+s+s2}{\PYZdq{}}\PY{p}{,} \PY{n}{hue}\PY{o}{=}\PY{l+s+s2}{\PYZdq{}}\PY{l+s+s2}{weekday}\PY{l+s+s2}{\PYZdq{}}\PY{p}{)}\PY{p}{;}
\end{Verbatim}


    \begin{center}
    \adjustimage{max size={0.9\linewidth}{0.9\paperheight}}{output_56_0.png}
    \end{center}
    { \hspace*{\fill} \\}
    
    We could attempt linear regression using \texttt{sns.lmplot} as shown
below, which hint at some relationships between temperature and
proportional casual, but the plot is still fairly unconvincing.

    \begin{Verbatim}[commandchars=\\\{\}]
{\color{incolor}In [{\color{incolor}29}]:} \PY{n}{sns}\PY{o}{.}\PY{n}{lmplot}\PY{p}{(}\PY{n}{data}\PY{o}{=}\PY{n}{bike}\PY{p}{,} \PY{n}{x}\PY{o}{=}\PY{l+s+s2}{\PYZdq{}}\PY{l+s+s2}{temp}\PY{l+s+s2}{\PYZdq{}}\PY{p}{,} \PY{n}{y}\PY{o}{=}\PY{l+s+s2}{\PYZdq{}}\PY{l+s+s2}{prop\PYZus{}casual}\PY{l+s+s2}{\PYZdq{}}\PY{p}{,} \PY{n}{hue}\PY{o}{=}\PY{l+s+s2}{\PYZdq{}}\PY{l+s+s2}{weekday}\PY{l+s+s2}{\PYZdq{}}\PY{p}{,} \PY{n}{scatter\PYZus{}kws}\PY{o}{=}\PY{p}{\PYZob{}}\PY{l+s+s2}{\PYZdq{}}\PY{l+s+s2}{s}\PY{l+s+s2}{\PYZdq{}}\PY{p}{:} \PY{l+m+mi}{20}\PY{p}{\PYZcb{}}\PY{p}{,} \PY{n}{height}\PY{o}{=}\PY{l+m+mi}{10}\PY{p}{)}
         \PY{n}{plt}\PY{o}{.}\PY{n}{title}\PY{p}{(}\PY{l+s+s2}{\PYZdq{}}\PY{l+s+s2}{Proportion of Casual Riders by Weekday}\PY{l+s+s2}{\PYZdq{}}\PY{p}{)}\PY{p}{;}
\end{Verbatim}


    \begin{center}
    \adjustimage{max size={0.9\linewidth}{0.9\paperheight}}{output_58_0.png}
    \end{center}
    { \hspace*{\fill} \\}
    
    A better approach is to use the "Local Smoothing" technique described in
lecture. We saw an example of this with cherry blossom race times. As a
reminder, the basic idea is that for each x value, we compute some sort
of representative y value that captures the data close to that x value.
One technique for local smoothing is "Locally Weighted Scatterplot
Smoothing" or LOWESS. An example is below. The red curve shown is a
smoothed version of the scatterplot.

    \begin{Verbatim}[commandchars=\\\{\}]
{\color{incolor}In [{\color{incolor}30}]:} \PY{k+kn}{from} \PY{n+nn}{statsmodels}\PY{n+nn}{.}\PY{n+nn}{nonparametric}\PY{n+nn}{.}\PY{n+nn}{smoothers\PYZus{}lowess} \PY{k}{import} \PY{n}{lowess}
         \PY{c+c1}{\PYZsh{} Make noisy data}
         \PY{n}{xobs} \PY{o}{=} \PY{n}{np}\PY{o}{.}\PY{n}{sort}\PY{p}{(}\PY{n}{np}\PY{o}{.}\PY{n}{random}\PY{o}{.}\PY{n}{rand}\PY{p}{(}\PY{l+m+mi}{100}\PY{p}{)}\PY{o}{*}\PY{l+m+mf}{4.0} \PY{o}{\PYZhy{}} \PY{l+m+mi}{2}\PY{p}{)}
         \PY{n}{yobs} \PY{o}{=} \PY{n}{np}\PY{o}{.}\PY{n}{exp}\PY{p}{(}\PY{n}{xobs}\PY{p}{)} \PY{o}{+} \PY{n}{np}\PY{o}{.}\PY{n}{random}\PY{o}{.}\PY{n}{randn}\PY{p}{(}\PY{l+m+mi}{100}\PY{p}{)} \PY{o}{/} \PY{l+m+mf}{2.0}
         \PY{n}{sns}\PY{o}{.}\PY{n}{scatterplot}\PY{p}{(}\PY{n}{xobs}\PY{p}{,} \PY{n}{yobs}\PY{p}{,} \PY{n}{label}\PY{o}{=}\PY{l+s+s2}{\PYZdq{}}\PY{l+s+s2}{Raw Data}\PY{l+s+s2}{\PYZdq{}}\PY{p}{)}
         
         \PY{c+c1}{\PYZsh{} Predict \PYZsq{}smoothed\PYZsq{} valued for observations}
         \PY{n}{ysmooth} \PY{o}{=} \PY{n}{lowess}\PY{p}{(}\PY{n}{yobs}\PY{p}{,} \PY{n}{xobs}\PY{p}{,} \PY{n}{return\PYZus{}sorted}\PY{o}{=}\PY{k+kc}{False}\PY{p}{)}
         \PY{n}{sns}\PY{o}{.}\PY{n}{lineplot}\PY{p}{(}\PY{n}{xobs}\PY{p}{,} \PY{n}{ysmooth}\PY{p}{,} \PY{n}{label}\PY{o}{=}\PY{l+s+s2}{\PYZdq{}}\PY{l+s+s2}{Smoothed Estimator}\PY{l+s+s2}{\PYZdq{}}\PY{p}{,} \PY{n}{color}\PY{o}{=}\PY{l+s+s1}{\PYZsq{}}\PY{l+s+s1}{red}\PY{l+s+s1}{\PYZsq{}}\PY{p}{)}
         \PY{n}{plt}\PY{o}{.}\PY{n}{legend}\PY{p}{(}\PY{p}{)}\PY{p}{;}
\end{Verbatim}


    \begin{center}
    \adjustimage{max size={0.9\linewidth}{0.9\paperheight}}{output_60_0.png}
    \end{center}
    { \hspace*{\fill} \\}
    
    In our case with the bike ridership data, we want 7 curves, one for each
day of the week. The x-axis will be the temperature and the y-axis will
be a smoothed version of the proportion of casual riders.

You should use
\href{http://www.statsmodels.org/dev/generated/statsmodels.nonparametric.smoothers_lowess.lowess.html}{\texttt{statsmodels.nonparametric.smoothers\_lowess.lowess}}
just like the example above. Unlike the example above, plot ONLY the
lowess curve. Do not plot the actual data, which would result in
overplotting. For this problem, the simplest way is to use a loop.

\textbf{Hints:} * Start by just plotting only one day of the week to
make sure you can do that first.

\begin{itemize}
\item
  lowess expects y coordinate first, then x coordinate.
\item
  Look at the top of this homework notebook for a description of the
  temperature field to know how to convert to fahrenheit. By default,
  the temperature field ranges from 0.0 to 1.0.
\end{itemize}

Note: If you prefer putting your plot in Celsius, that's fine as well!

    \begin{Verbatim}[commandchars=\\\{\}]
{\color{incolor}In [{\color{incolor}31}]:} \PY{k+kn}{from} \PY{n+nn}{statsmodels}\PY{n+nn}{.}\PY{n+nn}{nonparametric}\PY{n+nn}{.}\PY{n+nn}{smoothers\PYZus{}lowess} \PY{k}{import} \PY{n}{lowess}
         
         \PY{n}{plt}\PY{o}{.}\PY{n}{figure}\PY{p}{(}\PY{n}{figsize}\PY{o}{=}\PY{p}{(}\PY{l+m+mi}{10}\PY{p}{,}\PY{l+m+mi}{8}\PY{p}{)}\PY{p}{)}
         \PY{k}{for} \PY{n}{d} \PY{o+ow}{in} \PY{n+nb}{set}\PY{p}{(}\PY{n}{bike}\PY{p}{[}\PY{l+s+s1}{\PYZsq{}}\PY{l+s+s1}{weekday}\PY{l+s+s1}{\PYZsq{}}\PY{p}{]}\PY{p}{)}\PY{p}{:}
             \PY{n}{bike\PYZus{}day} \PY{o}{=} \PY{n}{bike}\PY{p}{[}\PY{n}{bike}\PY{p}{[}\PY{l+s+s1}{\PYZsq{}}\PY{l+s+s1}{weekday}\PY{l+s+s1}{\PYZsq{}}\PY{p}{]} \PY{o}{==} \PY{n}{d}\PY{p}{]}
             \PY{n}{ysmooth} \PY{o}{=} \PY{n}{lowess}\PY{p}{(}\PY{n}{bike\PYZus{}day}\PY{p}{[}\PY{l+s+s1}{\PYZsq{}}\PY{l+s+s1}{prop\PYZus{}casual}\PY{l+s+s1}{\PYZsq{}}\PY{p}{]}\PY{p}{,} \PY{p}{(}\PY{p}{(}\PY{n}{bike\PYZus{}day}\PY{p}{[}\PY{l+s+s1}{\PYZsq{}}\PY{l+s+s1}{temp}\PY{l+s+s1}{\PYZsq{}}\PY{p}{]}\PY{o}{*}\PY{l+m+mi}{41}\PY{p}{)}\PY{o}{*}\PY{p}{(}\PY{l+m+mi}{9}\PY{o}{/}\PY{l+m+mi}{5}\PY{p}{)}\PY{p}{)}\PY{o}{+}\PY{l+m+mi}{32}\PY{p}{,} \PY{n}{return\PYZus{}sorted}\PY{o}{=}\PY{k+kc}{False}\PY{p}{)}
             \PY{n}{g} \PY{o}{=} \PY{n}{sns}\PY{o}{.}\PY{n}{lineplot}\PY{p}{(}\PY{p}{(}\PY{p}{(}\PY{n}{bike\PYZus{}day}\PY{p}{[}\PY{l+s+s1}{\PYZsq{}}\PY{l+s+s1}{temp}\PY{l+s+s1}{\PYZsq{}}\PY{p}{]}\PY{o}{*}\PY{l+m+mi}{41}\PY{p}{)}\PY{o}{*}\PY{p}{(}\PY{l+m+mi}{9}\PY{o}{/}\PY{l+m+mi}{5}\PY{p}{)}\PY{p}{)}\PY{o}{+}\PY{l+m+mi}{32}\PY{p}{,} \PY{n}{ysmooth}\PY{p}{,} \PY{n}{label}\PY{o}{=}\PY{n}{d}\PY{p}{)}
         \PY{n}{g}\PY{o}{.}\PY{n}{set}\PY{p}{(}\PY{n}{xlabel}\PY{o}{=}\PY{l+s+s1}{\PYZsq{}}\PY{l+s+s1}{Temperature (Fahrenheit)}\PY{l+s+s1}{\PYZsq{}}\PY{p}{,} \PY{n}{ylabel}\PY{o}{=}\PY{l+s+s1}{\PYZsq{}}\PY{l+s+s1}{Casual Rider Proportion}\PY{l+s+s1}{\PYZsq{}}\PY{p}{)}
         \PY{n}{plt}\PY{o}{.}\PY{n}{title}\PY{p}{(}\PY{l+s+s2}{\PYZdq{}}\PY{l+s+s2}{Temperature vs Casual Rider Proportion by Weekday}\PY{l+s+s2}{\PYZdq{}}\PY{p}{)}\PY{p}{;}
         \PY{c+c1}{\PYZsh{} plt.legend();}
         \PY{c+c1}{\PYZsh{} bike.head()}
         \PY{c+c1}{\PYZsh{} bike\PYZus{}day}
         \PY{c+c1}{\PYZsh{} raise NotImplementedError()}
\end{Verbatim}


    \begin{center}
    \adjustimage{max size={0.9\linewidth}{0.9\paperheight}}{output_62_0.png}
    \end{center}
    { \hspace*{\fill} \\}
    
    \paragraph{Question 6c}\label{question-6c}

What do you see from the curve plot? How is \texttt{prop\_casual}
changing as a function of temperature? Do you notice anything else
interesting?

    \begin{Verbatim}[commandchars=\\\{\}]
{\color{incolor}In [{\color{incolor}32}]:} \PY{n}{q6c} \PY{o}{=} \PY{p}{(}\PY{l+s+s2}{\PYZdq{}}\PY{l+s+s2}{There is a general trend across all days that as temperature increases}\PY{l+s+s2}{\PYZdq{}}
         \PY{l+s+s2}{\PYZdq{}}\PY{l+s+s2}{so does the proportion of casual riders.}\PY{l+s+se}{\PYZbs{}n}\PY{l+s+s2}{ I also noticed that the Sat and Sun days}\PY{l+s+s2}{\PYZdq{}} 
         \PY{l+s+s2}{\PYZdq{}}\PY{l+s+s2}{are significantly higher proportion of casual riders.}\PY{l+s+se}{\PYZbs{}n}\PY{l+s+s2}{ This may suggest casual riders}\PY{l+s+s2}{\PYZdq{}}
         \PY{l+s+s2}{\PYZdq{}}\PY{l+s+s2}{use the bikes for leisure activity.}\PY{l+s+s2}{\PYZdq{}}\PY{p}{)}
\end{Verbatim}


    \subsection{Submission - IMPORTANT, PLEASE
READ}\label{submission---important-please-read}

For this assignment and future assignments (homework and projects) you
will also submit your free response and plotting questions to
gradescope. To do this, you can download as PDF
(\texttt{File-\textgreater{}Download\ As-\textgreater{}PDF\ via\ Latex\ (.pdf)}).
You are responsible for submitting and tagging your answers in
gradescope. For each free response and plotting question, please
include:

\begin{enumerate}
\def\labelenumi{\arabic{enumi}.}
\tightlist
\item
  Relevant code used to generate the plot or inform your insights
\item
  The written free response or plot
\end{enumerate}

We are doing this to make it easier on our graders and for you, in the
case you need to submit a regrade request. Gradescope (as of now) is
still better for manual grading.

    \subsection{Submission}\label{submission}

You're done!

Before submitting this assignment, ensure to:

\begin{enumerate}
\def\labelenumi{\arabic{enumi}.}
\tightlist
\item
  Restart the Kernel (in the menubar, select
  Kernel-\textgreater{}Restart \& Run All)
\item
  Validate the notebook by clicking the "Validate" button
\end{enumerate}

Finally, make sure to \textbf{submit} the assignment via the Assignments
tab in Datahub


    % Add a bibliography block to the postdoc
    
    
    
    \end{document}
